\documentclass{article}
\let\tldocenglish=1  % for live4ht.cfg
\usepackage{tex-live-zh-cn, indentfirst}
\usepackage{graphicx}
\usepackage{color}
\usepackage{xcolor}
\usepackage{verbatim}
\usepackage{comment}
\usepackage{minted}
\usepackage{lettrine}
\definecolor{bg}{rgb}{0.95,0.95,0.95}
% [tbd] means to be development
% [tbr] means to be view

\begin{document}

\title{%
  {\huge \textsf{Java 教程}\\\smallskip}%
  {\small \textit{翻译作品}}
}

\author{\textit{屈庆磊}\\[2mm]
        \email{quqinglei@icloud.com} 
       }

\date{\textit{2013 年 9 月 27 日}}

\maketitle
\begin{multicols}{2}
\tableofcontents
\end{multicols}

\section{说明}
\code{Java} 是一门面向对象的高级语言,最初由太阳微系统公司于1995年发布。\code{Java}运行于\code{Windows}
,\code{Mac OS},各种\code{UNIX}以及其变种的操作系统之上。本教程将会完整讲述\code{Java}语言。

本参考手册将会从简单、可练习的方式循序渐进的教你如何去学习这门语言。

\section{读者}
本参考手册准备向初学者讲述\code{Java}的基本概念和与其相关的高级用法。

\section{先决条件}
在你开始决定练习本参考手册上面的各种例子之前,我们假设你已经了解什么是计算机程序以及什么是编程语言。本参考
手册并不要求你有太多经验。

\section{版权声明}
本手册所有内容版权归属于\url{tutorialspoint.com} 在没有得到作者允许的情况下本文档的任何内容不得经过任何形式
的再发布,否则视为对侵犯版权法。

本手册可能不可避免的存在一些错误,我们不担保网站以及本手册内容的正确性。如果你发现网站以及本手册存在错误,
请发邮件至\email{webmaster@tutorialspoint.com}

\section{译者声明}
本文档为译者个人翻译作品,不允许任何形式的复制、再发布,如果需要可能会联系本作品的最初撰写者,取得中文版
的发布权利。

\section{\code{Java}观其大略}
\code{Java} 程序设计语言最初是由太阳微系统公司开发,起初是由\code{James Gosling} 创建,于1995年作为太阳
微系统公司的核心模块发布,也就是所谓的\code{Java 1.0 [J2SE]}

翻译这本手册的时候,\code{Java}的最新版本已经是\code{Java7u21}。归于\code{Java}语言的先进以及大众化,多
种平台被配置从而适合\code{Java}运行。\code{J2EE} 是\code{Enterprise Applications}的缩写,\code{J2ME} 是
\code{Mobile Applications}的缩写,即企业应用和移动应用。

太阳微系统公司已经把\code{J2}重命名为\code{Java SE} 为和\code{Java EE}以及\code{Java ME} 相互对照。\code{Java}
总冠以:编写一次,随处运行。

\code{Java}是:

\begin{itemize}
\item {\bf面向对象} \code{Java}的所有都是面向对象的。由于\code{Java}基于对象模型,所以她很容易扩展。
\item {\bf平台独立} 和其他包括\code{C/C++}在内的语言不通,\code{Java}不会被编译为机器相关的二进制程序,她
只是被编译为了字节码。字节码分布于网络中并为运行在平台上的虚拟机\code{(JVM)}所执行。
\item {\bf简单} \code{Java}被设计为简单易学的语言,如果你懂得简单的面向对象模型,就很容易掌握她。
\item {\bf安全} 基于\code{Java}的安全特性可以开发出免受病毒侵扰、无法篡改的系统,她基于公共密匙加密。
\item {\bf架构中立} \code{Java}编译器产生的是架构中立的对象文件,所以这种文件可以在许多处理器上运行。
\item {\bf可移植} 由架构中立、平台无关性的特征使\code{Java}非常容易移植。编译器和\code{Java}是采用\code{ANSI C}
编写,并属于\code{POSIX}的子集。
\item {\bf可靠性} \code{Java}努力在编译时和运行时做错误检查和运行时检查,当然错误无法避免。
\item {\bf多线程} \code{Java} 具有多线程编程的特点,可以同时运行多个任务。可以使开发者顺利运行交互性程序。
\item {\bf解释性} \code{Java} 边解释边执行,并不存储在任何地方。{\textsf The development process is more rapid
and analytical since the linking is an incremental and light weight process}
\item {\bf高性能} 使用\code{Just-In-Time}编译器使\code{Java}拥有高性能。
\item {\bf分布式} \code{Java}是为网络的分布式环境设计的。
\item {\bf动态性} \code{Java} 被认为比\code{C/C++}更为动态化,她被设计为更加适应不断变化的环境。她携带大量的
运行时信息,用于验证和解决运行时对象访问。
\end{itemize}

\subsection{\code{Java}的历史}
\code{James Gosling}在为他的机顶盒项目中于1991年创建了\code{Java}语言。起初她叫\code{Oak},那是因为在
\code{Gosling}的办公室外面有一棵橡树,她也曾经被称之为\code{Green}。后来被命名为\code{Java},这算是从
一列单词中随机选择的吧。

太阳微电子公司于1995年发布了\code{Java 1.0}版本,她承诺编写一次,随处运行。在流行的平台上提供无成本
运行时间。\textit{It promised Write once, Run Anywhere(WORA), providing no-cost run-times on popular platforms}

2007年5月8日,太阳微电子公司完成除了对一小部分不属于自己的源代码以外的\code{Java}所有核心代码的开源进程。

\subsection{你需要的工具}
你需要一个配置在\code{Pentium 200-MHz 64MB RAM}以上的计算机,来执行本教程中的例子,除此之外还需要一些软件:
\begin{itemize}
\item \code{Linux 7.1} 或者 \code{Wndows 95/98/2000/xp/vista/7} 操作系统
\item \code{Java JDK}
\item 一个编辑器,比如:\code{Microsoft Notepad}或者其他文本编辑器。
\end{itemize}
本文档会提供创建图形操作界面,网络,以及基于\code{Web}应用程序的必要技术。

\subsection{下一步,我们要干啥}
下一章节介绍开发环境的安装部署

\section{\code{Java}环境配置}
在我们为走更远的路之前,最重要的是先配置好环境。本章会告诉你如何下载、安装\code{Java},请跟随下面的步骤。
算了,我想这个章节还是不要翻译了,必经书本太老,\code{JDK}在不断更新,而且网络上可以搜的到,比如你可以搜:
\code{Java 开发环境搭建} 跟随网络上的教程就可以了。%[ommit]

\section{\code{Java} 基本语法}
我们可以把\code{Java}程序理解为定义好的一大堆对象,他们通过调用彼此的方法进行交互。我们现在简单的看一下类、
对象、方法和实例变量的意义。
\begin{itemize}
\item {\bf 对象} 对象拥有状态和行为。比如一只狗拥有一下状态特征:颜色、名字、品种,同样拥有摇头摆尾、狗吠、
吃饭等行为特征。对象为类的实例。
\item {\bf 类} 类可以描述为是一种模板或者蓝图,可以把它理解为是一种定义。
\item {\bf 实例变量} 每个对象都有其独立的实例变量。对象状态的建立其实就是实例变量的赋值。
\end{itemize}

\subsection{第一个程序}
我们看一下下面这段简单的代码,也就是我们在学习语言中遇到的第一个程序:\code{Hello World}。

\begin{minted}[linenos, frame=lines, framesep=2mm]{java}
// MyFirstJavaProgram.java
public class MyFirstJavaProgram {
	/* This is my first java program.
	 * This will print 'Hello World' as the output
	 */
	public static void main(String [] args) {
		System.out.println("Hello World"); // prints Hello World
	}
}
\end{minted}

保存以上代码到\dirname{MyFirstJavaProgram.java} 按照以下步骤编译并运行
\begin{itemize}
\item 打开一个文本编辑器
\item 保存以上代码为:\dirname{MyFirstJavaProgram.java}
\item 打开一个终端,切换到文件保存的路径,执行以下命令
\item \code{javac MyFirstJavaProgram.java}
\item \code{java MyFirstJavaProgram} 结果打印为:Hello World,则证明编译运行成功
\end{itemize}

\begin{minted}[linenos, frame=lines, framesep=2mm]{java}
# compile example LINUX
% javac MyFirstJavaProgram.java
% java MyFirstJavaProgram
Hello World
\end{minted}

\subsection{基本句法}
关于\code{Java}编程,下面的要点需要记住:
\begin{itemize}
\item {\bf 大小写敏感} \code{Java}是大小写敏感的语言,"Hello"和"hello"代表不同的意思
\item {\bf 类名} 类名的第一个字母要大些,当然你小写也可以,这只是一种约定
\item {\bf 方法名} 方法名开头字母应该为小写,这也属于编程的规范约定
\item {\bf 文件名} 文件名要和类名完全一致,除了后面的后缀以外
\item {\bf public static void main(String args[])} \code{Java} 程序从主函数开始执行
\end{itemize}

\subsection{\code{Java}标识符}
\code{Java}的组件包括各种名称,类名、变量、方法都被称为标识符。关于标识符的命名规范如下,其实和
其他语言类似,特别是类\code{C}语言。
\begin{itemize}
\item \code{Java}标识符由数字,字母和下划线,美元符号组成。在\code{Java}中是区分大小写的,而且还要
求首位不能是数字
\item \code{Java}关键字不能当作标识符
\item 合法标识符如:\code{age, \$salary, \_value, \_1\_value}
\item 不合法标识符如:\code{123abc, -salary}
\end{itemize}

\subsection{\code{Java}修饰符}
和其他语言一样\code{Java}同样拥有修饰符,通过修饰符我们可以更改类、方法等元素。基本上有两类修饰符:
\begin{itemize}
\item {\bf 访问修饰符}: default, public, protect, private
\item {\bf 不可访问修饰符}: final, abstract, strictfp [tbd]
\end{itemize}

未来的章节我们会对修饰符进行更深刻的讨论

\subsection{\code{Java} 变量}
\begin{itemize}
\item 局部变量
\item 类变量,(即静态变量)是全局变量
\item 常量,常量不能被更改,使用关键字\code{final}修饰,必须在常量声明时进行初始化
\end{itemize}

\subsection{\code{Java 数组}}
数组是同样类型的不同变量。\code{Java}的数组是存放在堆上的,这不同于\code{C},\code{C}默认是在栈上的。
我们会在后面的章节中讲述如何声明、创建、初始化数组。

\subsection{\code{Java} 枚举变量}
枚举变量从\code{Java 5.0}被引进。枚举限定一个变量有一个或者多个预定义的值,在此里列表中的值称之为枚举变量。

使用枚举可以尽量避免使用数字带来的错误。

下面是一个枚举类型的使用例子:[tbd]

\begin{minted}[linenos, frame=lines, framesep=2mm]{java}
class FreshJuice {
	enum FreshJuiceSize {SMALL, MEDUIM, LARGE}
	FreshJuiceSize size;
}

public classFreshJuiceTest {
	public static void main(String args[]) {
		FreshJuice juice = new FreshJuice();
		juice.size = FreshJuice.FreshJuiceSize.MEDUIM;
	}
}
\end{minted}

\subsection{保留字}
下面的表格中是\code{Java}的保留字,这些保留字不能被当做变量或者标识符名称。

\begin{center}
        \textit{
        \begin{tabular}{c|c|c|c}
        \hline
        \hline
        abstract & assert & boolean & break \\
        \hline
	byte & case & catch & char \\
	\hline
	class & const & continue & default \\
	\hline
	do & double & else & enum \\
	\hline
	extends & final & finally & float \\
	\hline
	for & goto & if & implements \\
	\hline
	import & instanceof & int & interface \\
	\hline
	long & native & new & package \\
	\hline
	private & protected & public & return \\
	\hline
	short & static & strictfp & super \\
	\hline
	switch & synchronized & this & throw \\
	\hline
	throws & transient & try &  void \\
	\hline
	volatile & while \\
	\hline
	\hline
        \end{tabular}
        }
\end{center}

\subsection{注释}
\code{Java}支持多行和单行注释,这个和\code{C/C++}一致,且注释不能嵌套。

\begin{minted}[linenos, frame=lines, framesep=2mm]{java}
public class MyFristJavaProgram {
	public static void main(String args[]) {
		/* 这里面包含的就是注释,编译器会自动忽略,
		   而且是多行注释 */
		System.out.println("Hello World"); // 单行注释,编译器自动忽略
	}
}
\end{minted}

\subsection{空白行}
空行和注释行都被称为空白行,\code{Java}会忽略他们

\subsection{继承}
如果你需要创建一个新的类型,而已经存在一个其他的类,你仅仅需要的是再添加一些其他的代码,你可以继承这个类。
从而演化成一个新的类。

这属于代码重用,避免你重复劳动。

\subsection{接口}
在\code{Java}语言中,我们可以定义接口,接口是两个对象之间用来通讯的方式。方法在继承中扮演很很重要的角色。

接口定义方法,派生类可以使用它,但是至于方法实现完全取决于派生类。

\subsection{下一步我们要干啥呢}
下一章节将要讲述对象和类。在下一章节的结尾你应该能对对象和类有一个清晰的认识。

\section{\code{Java} 对象和类}
\code{Java}是一种面向对象的编程语言。她支持以下的概念:
\begin{itemize}
\item {\bf Polymorphism} 多态性
\item {\bf Inheritance} 继承
\item {\bf Encapsulation} 封装
\item {\bf Abstraction} 抽象
\item {\bf Classes} 类
\item {\bf Objects} 对象
\item {\bf Instance} 实例
\item {\bf Method} 方法
\item {\bf Message Parsing} 消息解析
\end{itemize}

本章中我们将会讲述类和对象的概念。
\begin{itemize}
\item {\bf 对象} 对象拥有状态和行为。比如狗有颜色、名字、种类的状态,以及摇头摆尾、大声吠叫、吃饭等行为
\item {\bf 类} 类可以被理解成是模板或者蓝图,而对象是类的实例。或者说类是规范。
\end{itemize}

\subsection{对象}
让我们深入理解一下什么是对象,其实我们周围有很多很多的对象,车、狗、人等等,这些对象都拥有状态以及行为。

软件对象也拥有状态和行为,一个软件的对象状态存储在一定的字段中而行为在方法定义中。

所以在软件开发中,方法是在对象内部操作,对象间的通讯是通过方法之间的调用进行的。

\subsection{类}
我们可以把类看做是蓝图,下面是一个类的例子:

\begin{minted}[linenos, frame=lines, framesep=2mm]{java}
public class Dog {
	String breed; // 种类
	int age;
	String Color; 
	
	void barking() {
	}
	void hungry() {
	}
	void sleeping() {
	}
}
\end{minted}

一个类可以包含如下变量类型:

\begin{itemize}
\item {\bf 局部变量} 定义在方法、构造函数、代码块中的变量被称作局部变量,变量在方法里定义、初始化,并随
方法执行完毕而释放。
\item {\bf 常量} 常量是定义在类中,方法外的变量,这些变量在类加载的时候被初始化。常量可以被类中的方法、
构造函数、代码块访问。常量不允许修改,使用关键字\code{final}定义。
\item {\bf 类变量} 类变量在类中声明,在方法之外,使用\code{static}关键字定义,特点是如果一个类的对象更改
了类变量的值,那么在其他对象中,类变量的值随之改变。下面是个例子:

\begin{minted}[linenos, frame=lines, framesep=2mm]{java}
// Cat.java
class Cat {
	static String name = "Kitty";
	void sayName() {
		System.out.println(name);
	}
}

// CatTest.java
public class CatTest {
	public static void main(String args[]) {
		Cat cat01 = new Cat();
		Cat cat02 = new Cat();
		System.out.println("名字没被修改之前:");
		cat01.sayName();
		cat02.sayName();

		cat01.name = "Tom"; // 修改名字为Tom
		System.out.println("名字被修改之后:");
		cat01.sayName();
		cat02.sayName();
	}
}
// 运行后的结果:
名字没被修改之前:
Kitty
Kitty
名字被修改之后:
Tom
Tom
\end{minted}
\end{itemize}

\subsection{构造函数}
我们在谈论类的时候,构造函数是一个必然不会拉掉的话题,它是类的一个很重要的特性。每个类都有一个构造函数。
如果我们没有写构造函数,编译器会自动构造一个默认的构造函数。

构造函数必须和类名一致,每创建一个新的对象时,至少有一个构造函数将被调用。一个类可以有多个构造函数。

\begin{minted}[linenos, frame=lines, framesep=2mm]{java}
public class Puppy {
	public Puppy() {
	}
	public Puppy(String name) {
		//This constructor has one parameter, name.
	}

}
\end{minted}

\code{Java} 支持单例模式,你可以创建一个类的唯一实例,[tbr]

\subsection{创建一个对象}
我们在前面提及了类,类为对象提供蓝图。所以基本上一个对象是通过类创建的,在\code{Java}中\code{new}关键字
被用来创建一个新的对象。

下面是一个对象的创建步骤:
\begin{itemize}
\item {\bf 声明} 一个被声明为对象的变量
\item {\bf 实例化} 使用关键字\code{new}
\item {\bf 初始化} \code{new}关键字后面跟着一个构造器。这会初始化一个新的对象
\end{itemize}
下面是一个例子:

\begin{minted}[linenos, frame=lines, framesep=2mm]{java}
public class Puppy {
	public Puppy(String name) {
		System.out.println("Passed Name is: " + name);
	}
	public static void main(String args[]) {
		Puppy myPuppy = new Puppy("tommy");
	}
}
\end{minted}

如果我们编译并执行,结果会是:

\begin{minted}[linenos, frame=lines, framesep=2mm]{sh}
Passed Name is: tommy
\end{minted}

\subsection{访问实例的变量和方法}






























\end{document}














