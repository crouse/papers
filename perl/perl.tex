\documentclass{article}
\let\tldocenglish=1  % for live4ht.cfg
\usepackage{tex-live-zh-cn, indentfirst}
\usepackage{graphicx}
\usepackage{color}
\usepackage{xcolor}
\usepackage{verbatim}
\usepackage{comment}
\usepackage{minted}
\definecolor{bg}{rgb}{0.95,0.95,0.95}

\begin{document}

\title{%
  {\huge \textsf{Perl}\\\smallskip}%
  {\small \textit{翻译作品,来自perl官网}}
}

\author{\textit{屈庆磊}\\[2mm]
        \email{quqinglei@icloud.com} 
       }

\date{\textit{2013 年 8 月 29 日}}

\maketitle
\begin{multicols}{2}
\tableofcontents
\end{multicols}

\section{说明}
文中所有带[tbd] 字样的地方说明有疑问,还未曾解决!

\section{Perl 正则快速入门}
\subsection{名字}
本文讲述perl正则表达式如何快速入门,基本采用例子讲解
没有完全按照原文进行翻译,主要是使用其中的例子。
\url{http://perldoc.perl.org}
\subsection{描述}
本文包含了Perl语言的最基本的正则表达式理解、创建、使用
\subsection{指导}

\subsubsection{单词匹配}
最简单的正则表达式其实就是一个单词,或者多个单词,由多个字符组成的字符串。含有一个单词的正则表达式可以匹配 包含
此单词的任意字符串。

\begin{minted}[linenos, frame=lines, framesep=2mm]{perl}
"Hello World" =~ /World/; # 匹配
\end{minted}

在上面的语句中,\code{{\color{blue}=~}}是匹配操作符,\code{{\color{blue}//}}表示包含的是模式串。可以理解为,字符串
\code{"Hello World"} 中是否有模式\code{World},如果有则他们匹配,如果把以上语句赋值给一个变量,那么如果匹配,变量
值为1,反之为空\footnote{注意,如果不匹配得到的结果并不是0,而应该是个空字符串[tbd]}

下面的语句很简单,就不一一细说了,注意符号\code{{\color{blue}=~}} 为不匹配符。还有第四行和第五行其实就是说明了一
个变量替换的道理,也就是说正则也是可以用变量替换表达的。

\begin{minted}[linenos, frame=lines, framesep=2mm]{perl}
print "It matches\n" if "Hello World" =~ /World/;
print "It doesn't match\n" if "World Hello" !~ /Hello/;

$greeting = "Adam";
print "It matches\n" if "Hello Adam" =~ /$greeting/;
\end{minted}

如果你要匹配:\code{{\color{blue}\$\_}} ,就不需要:\code{{\color{blue}\$\_ =~ }} 了。看下面代码:

\begin{minted}[linenos, frame=lines, framesep=2mm]{perl}
$_ = "Hello World";
print "It matches\n" if /World/;
\end{minted}

模式匹配符 \code{{\color{blue}//}} 也是可以被替换成其他任意的符号,只需要在分隔符号前面加一个:
\code{{\color{blue}'m'}} 即可,如下所示:

\begin{minted}[linenos, frame=lines, framesep=2mm]{perl}
"Hello World" =~ m!World!; # 匹配,使用'!'作为分隔符
"Hello World" =~ m{World}; # 匹配,使用'{}' 包含模式匹配
"/usr/bin/perl" =~ m"/perl" # '/' 变成了一般符号
\end{minted}

正则表达式必须精确匹配字符串的一部分,才能保证此语句为真,空格也是字符!

\begin{minted}[linenos, frame=lines, framesep=2mm]{perl}
"Hello World" =~ /world/; # 不匹配,大小写问题
"Hello World" =~ /o W/; # 匹配, ' ' 也是一般字符
"Hello World" =~ /World /; # 不匹配,字符串后面没有 ' '(空格)
\end{minted}

Perl 会先匹配字符串前面的匹配点,也就是说从字符串的开头开始匹配,一旦匹配就返回真:

\begin{minted}[linenos, frame=lines, framesep=2mm]{perl}
"Hello World" =~ /o/; # 它匹配的其实是 'Hello' 中的 'o'
"That hat is red" =~ /hat/; # 其实它匹配的是 'That' 中的 'hat' 字符串
\end{minted}

在Perl中,并不是所有的字符就可以直接的匹配,有些字符属于{\color{blue}元字符}(metacharacters)\footnote{所谓元字符
就是指那些在正则表达式中具有特殊意义的专用字符,可以用来规定其前导字符(即位于元字符前面的字符)在目标对象中的出
现模式}需要转义后才可以匹配。下面给出了正则表达式中的元字符:

\begin{minted}[linenos, frame=lines, framesep=2mm]{perl}
{}[]()^$.|*+?\
\end{minted}

在元字符前面使用反斜杠 \code{{\color{blue}\\}} 转义就可以匹配啦,如下所示:

\begin{minted}[linenos, frame=lines, framesep=2mm]{perl}
"2+2=4" =~ /2+2/; # 不匹配,原因是'+'属于元字符
"2+2=4" =~ /2\+2/; # 匹配,我们把'+'给转义成为了普通字符
'C:\WIN32' =~ /C:\\WIN/; # 匹配,我们使用反斜杠把反斜杠给转义成为了普通字符
"/usr/bin/perl" =~ /\/usr\/bin\/perl/; # 匹配,不解释了,可以参考下面这个语句

# 当我们使用其他字符作为模式匹配符号时,就不需要对 '/' 进行转义了,它只是一个
# 模式匹配字符,并不是元字符
print "matches!\n" if "/usr/bin/perl" =~ m{/usr/bin/perl}; 
\end{minted}

不可打印字符可以使用转义序列表示,如:\code{{\color{blue}\\t}} 可以表示为制表符 \code{{\color{blue}\\n}} 可以表示
为换行符\code{{\color{blue}\\r}} 可以表示为回车。另外任意字节都可以通过{\color{red}八进制或者十六进制}数字的转
义序列表示,下面一些例子:

\begin{minted}[linenos, frame=lines, framesep=2mm]{perl}
"1000\t2000" =~ m(0\t2); # 匹配
"cat" =~ /\143\x61\x74/ # 匹配,其中'143' 是c的八进制ASCII,x61、x74 分别为a、t的十六进制ASCII码
\end{minted}

可以使用变量替换正则表达式:

\begin{minted}[linenos, frame=lines, framesep=2mm]{perl}
$foo = 'house';
'cathouse' =~ /cat$foo/; # 使用$foo变量替换,匹配
'housecat' =~ /${house}cat/; # 匹配
\end{minted}

废话就不翻译了,我们可以使用锚点来确定匹配位置,比如使用:\code{\^} 来表示匹配字符串开头,使用
\code{{\color{blue}\$}} 来匹配字符串结尾。如下代码所示:

\begin{minted}[linenos, frame=lines, framesep=2mm]{perl}
"housekeeper" =~ /keeper/; # 匹配
"housekeeper" =~ /^keeper/; # 不匹配
"housekeeper" =~ /keeper$/; # 匹配
"housekeeper" =~ /^housekeeper/; # 匹配
\end{minted}

\subsubsection{使用字符类}
\textit{
字符类可以表示一类字符,我们还是看例子吧,这样比较清楚
}

\begin{minted}[linenos, frame=lines, framesep=2mm]{perl}
/cat/; # 匹配cat
/[bcr]at/; # 匹配bat cat rat
"abc" =~ /[cab]/; 先匹配到 a
\end{minted}

在上面代码中的最后一样,匹配时最先匹配到的其实是 'a' 虽然说它在正则表达式中的第二个,我们以为先匹配到的应该
是c,但是,perl在处理此问题的时候对[ ]里面的字符进行了排序。

\begin{minted}[linenos, frame=lines, framesep=2mm]{perl}
/[yY][eE][sS]/; # 不区分大小写匹配 'yes'
/yes/i; # 正则后面多了个 i 就可以代替上面一行代码的功能了!
\end{minted}

在上面代码中最后一行,我们使用了一个字符\code{{\color{blue}i}} 它命令正则不区分大小写进行匹配。




































\end{document}

