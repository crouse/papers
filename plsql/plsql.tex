\documentclass{article}
\let\tldocenglish=1
\usepackage{tex-live-zh-cn, indentfirst}
\usepackage{graphicx}
\usepackage{color}
\usepackage{xcolor}
\usepackage{verbatim}
\usepackage{comment}
\usepackage{minted}
\definecolor{bg}{rgb}{0.95,0.95,0.95}

%\begin{minted}{python}
%\end{minted}
%\colorbox{blue}{text}
%\textcolor{blue}{GEO}

\begin{document}

\title{
  {\huge \textsf{PL/SQL 简明教程}\\\smallskip}
  {\small \textit{翻译作品}}
}

\author{\textit{屈庆磊}\\[2mm]
        \email{quqinglei@icloud.com} 
       }

\date{\textit{2013 年 7 月}}

\maketitle
\begin{multicols}{2}
\tableofcontents
\end{multicols}

\section{关于此文档}
\subsection{PL/SQL 简明教程}
PL/SQL 是添加了过程特性的一门编程语言,是甲骨文公司于九十年代初期开发的对SQL的增强实现。

PL/SQL 是甲骨文数据库系统的三个关键点之一,其余的两个点为SQL本身,以及JAVA。

本文档将会带给你一个对PL/SQL以及关系型数据库,比较容易理解的学习过程。

\subsection{面向读者}
本文档面向有熟练编程经验的专家,提供简单的学习方式。本文档会给你提供一个很容易理解的学习方式,当完成本文档
的学习后,你有一个中级的能力。

\subsection{先决条件}
在看本书的时候,你需要对计算机科学有一定的了解,比如基本的软件概念,如什么是数据库,源代码,文本编辑器,程序运行
方式,当然如果你懂SQL并且有其他语言的编程经验,那你可以比较轻松的学习了。

\subsection{版权声明}
\copyright 本文档里的所有内容、图片皆为\url{tutorialspoint.com}所有。来自此网站的任何内容或者本文档的内容,在没有
经过同意下不允许任何形式的再发行。如果不如此做则违法版权法。

此文档可能存在不准确或者错误,本网站并不承担任何担保,如果你找到了错误请联系\email{webmaster@tutorialspoint.com}


\section{PL/SQL 总览}
\textit{本章节介绍PL/SQL的基本概念和定义}
\subsection{简介}
PL/SQL 语言是甲骨文公司于二十世纪八十年代末发明的一种关系型数据库,下面是PL/SQL的一些可圈点的特点:
\begin{itemize}
\item 可移植、高性能的事务性处理语言
\item 提供独立于操作系统的内建语言环境
\item 可以直接通过命令行调用
\item 可以直接通过外部程序调用
\item 语法采用ADA和Pascal的语法格式
\item 此语言不只是只适合甲骨文数据库,同样适用于TimesTen 内存数据库以及IBM DB2
\end{itemize}

\subsection{特性}
\begin{itemize}
\item 和SQL紧密关联
\item 提供广泛的错误检查
\item 提供很多的数据类型
\item 提供结构化编程特性
\item 提供函数以及过程调用
\item 支持面向对象的编程
\item 支持网页程序和服务器编程
\end{itemize}

\subsection{优势}
\begin{itemize}
\item SQL是一种标准数据库语言,PL/SQL 和SQL紧密结合。支持静态和动态的SQL。支持数据操纵、事务控制。
动态SQL是允许嵌入PL/SQL块中的数据定义语句。
\item 支持对代码块的一次整块发送,避免网络拥堵提供程序高性能。
\item 提供程序员高效的操作
\item 让你少花费很多时间考虑一些其他的东西
\item 提供很高的安全性
\item 提供对已定义包的直接调用
\item 提供面向对象的编程
\item 提供对网络程序和服务器程序的支持
\end{itemize}

\section{基本语法}
\textit{本章节描述一些基本的语法规则}
\subsection{范例}
\textit{PL/SQL 是一种结构化语言,这意味着用它写的程序可以划分为很多程序块,每个块包含以下三个子块}

\begin{description}
\item[定义] 本块开始于关键字\textbf{DECLARE}。是一个可选择的模块,定义所有的变量、游标、子程序、以及其他元素。
\item[可执行命令] 本块开始于\textbf{BEGIN}结束于\textbf{END} 是一个强制性必须要有的模块。至少要有一行代码,哪怕是
	一个NULL命令。
\item[异常处理] 本块开始于关键字\textbf{EXCEPTION} 此块是可选择的,可有可无。用来处理错误和异常。
\end{description}


每一个语句要以分号结束,PL/SQL语句可以通过\textbf{BEGIN END} 嵌套于其他语句里,下面是一个很基本的PL/SQL 模块:

\begin{minted}{sql}
DECLARE 
	<declarations section> 
BEGIN 
	<executable command(s)> 
EXCEPTION 
	<exception handling> END;
\end{minted}

\subsection{HelloWorld}
\begin{minted}{sql}
DECLARE
	message varchar2(20) := 'Hello, World!';
BEGIN
	dbms_output.put_line(message);
END;
/
\end{minted}

\subsection{标识符}
PL/SQL 标识符包含常量、变量、异常、过程、游标和一些保留字。标识符是由字符、特殊符号、美元符号等符号组成。
标识符的长度不要大于30个字符。这东西我就不多说了,对于有编程经验的人意义不是很大,因为都是相通的。

\subsection{分隔符}
\begin{description}
\item[$+,-,*,/$] 加、减、乘、除
\item[$\%$] 属性指标
\item[$'$] 字符串分隔符
\item[$.$] 元件选择
\item[$(,)$] 表达式或者是列表分割
\item[$:$] 主变量指示符
\item[$,$] 分隔符
\item[$"$] 引用标识符分隔符
\item[$=$] 关系操作符
\item[$@$] 远程访问操作符
\item[$;$] 语句终结符
\item[$:=$] 赋值操作符
\item[$=>$] Association operator
\item[$||$] 字符串连接符
\item[$**$] Exponentiation operator
\item[$<<,>>$] Label delimiter (begin and end)
\item[$/*,*/$] 多行注释
\item[$--$] 单行注释
\item[$..$] 范围
\item[$<,>, <=, >= $] 关系操作符
\item[$<>, '=, ~=, ^=$] 不等号的其他版本
\end{description}

\subsection{注释}
废话少说,就是注释,注释就是把一些说明性的文档用某些约定的符号注释上,编译器会自动过滤掉此类语句,下面是个
例子

\begin{minted}{sql}
DECLARE
	-- variable declaraton 其实这就是注释啦,此行不会执行
	message varchar2(20) := 'Hello, World!';
BEGIN
	/*
		PL/SQL executable statement(s)
		这里面就是多行注释,你可以写很多行说明的
	*/
	dbms_output.put_line(message);
END;
/
\end{minted}

当以上代码执行完后,会有以下输出:

\begin{minted}{sql}
Hello, World

PL/SQL procedure successfully completed.
\end{minted}

\subsection{PL/SQL 程序单元}
一个PL/SQL程序单元可能是以下任何项目:\textit{其实这些都是废话,还是看实例比较好}

\begin{itemize}
\item PL/SQL 块
\item 函数
\item 包
\item 包体
\item 过程
\item 触发器
\item 类型
\item 类型块
\end{itemize}

以上所有的项目都会在下面的讲解中提到

\section{数据类型}
\textit{PL/SQL 变量、常量、参数必须拥有合法的数据类型,废话我就不翻译了
}

\begin{tabular}{c|c}
\hline
种类 & 说明 \\
\hline
标量 & 单一值,无内部结构,如NUMBER, DATE, BOOLEAN \\
\hline
大对象(LOB) & 比如文本、图像、视频剪辑、声音文件 \\
\hline
混合变量 & 有内部结构的变量,如集合与记录 \\
\hline
参考 & 指向其它数据的东东(tbd) \\
\hline
\end{tabular}





















\end{document}

