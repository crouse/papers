\documentclass{article}
\let\tldocenglish=1  % for live4ht.cfg
\usepackage{tex-live-zh-cn, indentfirst}
\usepackage{graphicx}
\usepackage{color}
\usepackage{xcolor}
\usepackage{verbatim}
\usepackage{comment}
\usepackage{minted}
\definecolor{bg}{rgb}{0.95,0.95,0.95}

\begin{document}

\title{%
  {\huge \textsf{Scribe 编译与非超级权限运行}\\\smallskip}%
  {\small \textit{关于编译Scribe需要注意的问题}}
}

\author{\textit{屈庆磊}\\[2mm]
        \email{quqinglei@icloud.com} 
       }

\date{\textit{2013 年 8 月 11 日}}

\maketitle
\begin{multicols}{2}
\tableofcontents
\end{multicols}


\section{编译须知}
\textit{
Scribe 依赖一些第三方的包,所以在编译的时候需要一些预先配置,官网上所说的编译过程并不全面,下面是一个较为全面的编译方法。但仍然存在需要改进的地方。
}

\begin{description}
\item[libevent] libevent是一个事件触发的网络库,适用于windows、linux、bsd等多种平台,内部使用select、epoll、kqueue等系统调用管理事件机制。
著名分布式缓存软件memcached也是libevent based,而且libevent在使用上可以做到跨平台,而且根据libevent官方网站上公布的数据统计,似乎也有着非凡的性能。
\item[boost] Boost库是一个可移植、提供源代码的C++库,作为标准库的后备,是C++标准化进程的发动机之一。 Boost库由C++标准委员会库工作组成员发起,
其中有些内容有望成为下一代C++标准库内容。在C++社区中影响甚大,是不折不扣的“准”标准库。Boost由于其对跨平台的强调,对标准C++的强调,与编写平
台无关。大部分boost库功能的使用只需包括相应头文件即可,少数(如正则表达式库,文件系统库等)需要链接库。但Boost中也有很多是实验性质的东西,
在实际的开发中实用需要谨慎。boost 在一些播放软件和音效中指增强,比如Bass Boost,低音增强。
\item[thrift] 是一个软件框架,用来进行可扩展且跨语言的服务的开发。它结合了功能强大的软件堆栈和代码生成引擎
\end{description}

\section{本例中下载的包}
\begin{description}
\item[boost\_1\_54\_0.tar.bz2] boost 库,可以从其官网下载,地址为:\url{www.boost.org/}
\item[thrift-0.9.0.tar.gz] thrift 包,官网地址:\url{http://thrift.apache.org/}
\item[scribe-master.zip] scribe 包,可以从github上下载,地址为:\url{https://github.com/facebook/scribe/}
\end{description}

\section{编译并安装}
\textit{
不建议在配置时使用自定义的安装路径,这样在编译的时候可能会带来一些不必要的困难和错误。
}

\subsection{安装编译依赖包}
\textit{
基本的编译环境需要的依赖包:
}

\begin{minted}{sh}
yum install automake flex bison libevent-devel #libevent-devel 不属于基本开发包,但必须安装
\end{minted}


\section{安装boost}
\textit{
不建议设置boost的安装路径,采用默认的安装方式反而更容易编译scribe。
boost 默认情况下会把boost的动态库安装到\dirname{/usr/local/lib} 所以
我们在编译完成后可以直接把它拷贝到我们的路径下,因为二进制文件在运行时
只需要boost的动态链接库,而不需要头文件和代码。
}

\begin{minted}{sh}
# tar xjvf boost_1_54_0.tar.bz2
# cd boost_1_54_0
# ./bootstrap.sh
# ./b2
# ./b2 install
\end{minted}

\section{安装thrift和其下的fb303}
\textit{
thrift 的编译需要boost的支持,所以我们在它之前安装了boost,下面是编译步骤。
}

\begin{minted}{sh}
# tar xzvf thrift-0.9.0.tar.gz
# cd thrift-0.9.0
# ./configure
# make
# make install

# cd contrib/fb303
# ./bootstrap.sh
# make
# make install
\end{minted}

\section{编译scribe}
\textit{
我们在前面编译并安装一些依赖的时候全部采用默认方式,这样在编译scribe的时候就能省一些不必要的麻烦,下面是步骤。
}

\begin{minted}{sh}
# ./bootstrap.sh
# ./configure
# make # 在编译的时候就可能会出错,这是因为它设置的依赖库的原因,总之它存在编译的bug。

# 可以到src路径直接使用命令编译
# g++  -Wall -O3 -L/usr/local/lib/ -lboost_system -lboost_filesystem  \
	-o scribed store.o store_queue.o conf.o file.o conn_pool.o scribe_server.o \
	network_dynamic_config.o dynamic_bucket_updater.o  env_default.o \
	 -L/usr/local/lib -L/usr/local/lib -L/usr/local/lib -lfb303 -lthrift \
	 -lthriftnb -levent -lpthread  libscribe.a libdynamicbucketupdater.ae.a \
	libdynamicbucketupdat
\end{minted}

编译完成后会生成一个可执行的scribed文件,它就是我们想要的东西!

\section{非超级权限运行}
\textit{
绝大多数应用软件是不需要超级用户权限运行的,scribe并不例外,我们没必要采用静态编译的方式解决问题,导出一个库的环境变量
就解决了这个问题。我们把scribed依赖的动态库放到一个路径,并导出这个路径的环境变量,它就可以使用了,而每个普通用户也维护着一份
环境变量,所以我们可以把环境变量写在脚本上,在程序执行之前导出。
}

\section{打到一个文件夹里}
\textit{
为了方便使用,我们把scribed打到一个文件夹里,方法如下:
}

\begin{minted}{sh}
# mkdir -p scribe-bin/bin/
# mkdir -p scribe-bin/lib/
# cp -r /usr/local/lib/* scribe-bin/lib/ 
# cp scribed scribe-bin/bin/ # 把生成的二进制文件放到文件夹的bin路径
# tar czvf scribe-bin.tar.gz scribe-bin

\end{minted}

\section{环境变量}
\textit{
以下环境变量是用来告诉程序去哪里找动态链接库的东西。
}

\begin{minted}{sh}
export LD_LIBRARY_PATH="/usr/lib:path-to-scribe-bin/lib"
\end{minted}

\end{document}













