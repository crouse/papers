\documentclass{article}
\let\tldocenglish=1  % for live4ht.cfg
\usepackage{tex-live-zh-cn, indentfirst}
\usepackage{graphicx}
\usepackage{color}
\usepackage{xcolor}
\usepackage{verbatim}
\usepackage{comment}
\usepackage{minted}
\definecolor{bg}{rgb}{0.95,0.95,0.95}

\begin{document}

\title{%
  {\huge \textsf{oracle notes}\\\smallskip}%
  {\small \textit{关于oracle的笔记}}
}

\author{\textit{屈庆磊}\\[2mm]
        \email{quqinglei@icloud.com} 
       }

\date{\textit{2013 年 5 月}}

\maketitle
\newpage
\begin{multicols}{2}
\tableofcontents
%\listoftables
\end{multicols}
\newpage 

\section{常识}
\begin{itemize}
\item[(1)]
在\code{windows}下安装完毕后,可以在命令提示符里输入:
\code{sqlplus "/as sysdba"} 可直接进入其\code{shell}。

\item[(2)]
如果需要在命令提示符内执行某个脚本:
脚本文件位置:\dirname{D://sqlscripts/test.sql}
执行步骤:\code{SQL> @D://sqlscripts/test.sql}

\item[(3)]
打开服务端的标准输出命令:
\code{SQL> set serveroutput on}
\end{itemize}

\section{常用函数}
\code{nvl(x, 0)} 如果x为空,则替换为0

\section{常用语句}
\subsection{JOIN}
\begin{minted}{sql}
-- create tc td
create tc(id number, name varchar2(20), address varchar2(50));
create td(id number, sex varchar2(5), job varchar2(50));

-- insert data to tc td
insert into tc values(1, '张三', '北京市昌平区');
insert into td values(1, '男', '程序员');

-- query from tc td
select tc.id, tc.name, tc.address, td.sex, td.job from tc left join
td on td.id = tc.id;
\end{minted}

\section{存储过程}
\subsection{简单的例子}
\subsubsection{匿名块的例子}
\begin{minted}{sql}
--hello.sql
declare 
	message varchar2(20) := 'Hello world';
begin
	dbms_output.put_line(message);
end;
/
\end{minted}
\subsubsection{标准的存储过程}
\begin{minted}{sql}
--hello.sql
create or replace procedure test
as
	message varchar2(20) := 'Hello world';
begin
	dbms_output.put_line(message);
end test;
\end{minted}

\subsection{变量声明}
\begin{minted}{sql}
declare
	num1 integer;
	num2 real;
	num3 double precision;
begin
	null;
end;
/
\end{minted}

\subsection{自定义类型}
\begin{minted}{sql}
declare 
	subtype name IS char(20); 
	subtype message IS varchar2(100);
	salutation name;
	greeting message;
begin
	salutation := 'Reader ';
	greeting := 'Welcome to the World of PL/SQL';
	dbms_output.put_line('Hello ' || salutation || greetings);
end
/
\end{minted}

\subsection{变量初始化}
\begin{minted}{sql}
declare
	a integer := 10;
	b integer := 20;
	c integer;
	f real;
begin
	c := a + b;
	dbms_output.put_line('Value of c: ' || c);
	f := 70.0/3.0;
	dbms_output.put_line('Value of f: || f);
end;
/
\end{minted}

\subsection{局部变量和全局变量}
\begin{minted}{sql}
declare
	-- Global variables
	num1 number := 95;
	num2 number := 85;
begin
	dbms_output.put_line('Outer Variable num1: ' || num1);
	dbms_output.put_line('Outer Variable num2: ' || num2);
	--<
	declare
		-- Local variables
		num1 number := 195;
		num2 number := 185;
	begin
		dbms_output.put_line('Inner Variable num1: ' || num1);
		dbms_output.put_line('Inner Variable num2: ' || num2);
	end;
	-->
end;
/
\end{minted}

\subsection{指定查询结果为变量的值}
\textsf{返回到\code{if02}处}~\ref{call01}
\label{table01}
\begin{minted}{sql}
-- create a table
create table customers (
	ID int not null,
	name varchar(20) not null,
	age int not null,
	address char(25),
	salary decimal(18, 2),
	primary key (ID)
);

-- insert values to this table
insert into customers (id,name,age,address,salary)
values (1, 'Ramesh', 32, 'Ahmedabad', 2000.00 );
insert into customers (id,name,age,address,salary)
values (2, 'Khilan', 25, 'Delhi', 1500.00 );
insert into customers (id,name,age,address,salary)
values (3, 'kaushik', 23, 'Kota', 2000.00 );
insert into customers (id,name,age,address,salary)
values (4, 'Chaitali', 25, 'Mumbai', 6500.00 );
insert into customers (id,name,age,address,salary)
values (5, 'Hardik', 27, 'Bhopal', 8500.00 );
insert into customers (id,name,age,address,salary)
values (6, 'Komal', 22, 'MP', 4500.00 );

-- procedure
declare
	c_id customers.id%type := 1;
	c_name customer.name%type;
	c_addr customer.address%type;
	c_sal customer.salary%type;
begin
	select name, address, salary into c_name, c_addr, c_sal
	from customers
	where id = c_id;
	
	dbms_output.put_line
	('Customer ' || c_name || ' from ' || c_addr || ' earns ' || c_sal);
end;
/
\end{minted}

\subsection{声明常量}
\begin{minted}{sql}
declare
	-- constant declaration
	pi constant number := 3.141592654;
	-- other declaration
	radius number(5, 2);
	dia number(5, 2);
	circumference number(7, 2);
	area number(10, 2);
begin
	-- processing
	radius := 9.5;
	dia := radius * 2;
	circumference := 2.0 * pi * radius;
	area := pi * radius * radius;
	-- output
	dbms_output.put_line('Radius: ' || radius);
	dbms_output.put_line('Diameter: ' || dia);
	dbms_output.put_line('Circumference: ' || circumference);
	dbms_output.put_line('Area: ' || area);
end;
/
\end{minted}

\subsection{判断语句}
\begin{minted}{sql}
declare
	a number (2) := 21;
	b number (2) := 10;
begin
	if (a = b) then
		dbms_output.put_line('Line 1: a is equal to b');
	else
		dbms_output.put_line('Line 1: a is not equal to b');
	end if;

	if (a < b) then
		dbms_output.put_line('Line 2: a is less than b');
	else
		dbms_output.put_line('Line 2: a is not less than b');
	end if;

	if (a > b) then
		dbms_output.put_line('Line 3: a is greater than b');
	else
		dbms_output.put_line('Line 3: a is not greater than b');
	end if;

	-- Lets change value of a and b
	a := 5;
	b := 20;
	if (a <= b) then
		dbms_output.put_line('Line 4: a is either equal or less than b');
	end if;
		
	if (b >= a) then
		dbms_output.put_line('Line 5: b is either equal or greater than a');
	end if;

	if (a <> b) then
		dbms_output.put_line('Line 5: a is not equal to b');
	end if;
end;
/
\end{minted}

\subsection{\code{LIKE}}
\begin{minted}{sql}
declare procedure compare (value varchar2, pattern varchar2) is
begin
	if value like pattern then
		dbms_output.put_line('True');
	else
		dbms_output.put_line('False');
	end if;
end;

begin
	compare('Zara Ali', 'Z%A_i');
	compare('Nuha Ali', 'Z%A_i');
end;
/
\end{minted}

\subsection{\code{BETWEEN}}
\begin{minted}{sql}
declare
	x number(2) := 10;
begin
	if (x between 5 and 20) then
		dbms_output.put_line('True');
	else
		dbms_output.put_line('False');
	end if;
\end{minted}

\subsection{\code{IN and IS NULL}}
\begin{minted}{sql}
declare
	letter varchar2(1) := 'm';
begin
	if (letter in ('a', 'b', 'c')) then
		dbms_output.put_line('True');
	else
		dbms_output.put_line('False');
	end if;

	if (letter in ('m', 'n', 'o')) then
		dbms_output.put_line('True');
	else
		dbms_output.put_line('False');
	end if;

	if (letter is null) then
		dbms_output.put_line('True');
	else
		dbms_output.put_line('False');
	end if;
end;
/
\end{minted}

\subsection{\code{Logic Operators: and/or/not}}
\begin{minted}{sql}
declare
	a boolean := true;
	b boolean := false;
begin
	if (a and b) then
		dbms_output.put_line('Line 1: Condition is true');
	end if;

	if (a or b) then
		dbms_output.put_line('Line 2: Condition is true');
	end if;

	if (not a) then
		dbms_output.put_line('Line 3: a is not true');
	end if;

	if (not b) then
		dbms_output.put_line('Line 4: b is not true');
	else
		dbms_output.put_line('Line 4: b is true');
	end if;

\end{minted}

\subsection{算术操作}
\begin{minted}{sql}
declare 
	a number(2) := 20;
	b number(2) := 10;
	c number(2) := 15;
	d number(2) := 5;
	e number(2);

begin
	e := (a + b) * c / d;	--(30 * 15) / 5
	dbms_output.put_line('Value of (a + b) * c / d is: ' || e);

	e := ((a + b) * c) / d;	--(30 * 15) / 5
	dbms_output.put_line('Value of ((a + b) * c) / d is: ' || e);

	e := (a + b) * (c / d);	-- (30) * (15 / 5)
	dbms_output.put_line('Value of (a + b) * (c / d) is: ' || e);

	e := a + (b * c) / d;	-- 20 + (150 / 5)
	dbms_output.put_line('Value of a + (b * c) / d is: ' || e);

end;
/
\end{minted}

\subsection{条件}
\subsubsection{\code{if-then-endif}}
\begin{minted}{sql}
declare
	a number(2) := 10;
begin
	a := 10;
	-- check the boolean condition using if statement
	if (a < 20) then
		dbms_output.put_line('a is less than 20');
	end if;
	dbms_output.put_line('value of a is: ' || a);
end;
/
\end{minted}

\subsubsection{\code{if02}}
\textsf{关于表请参考~\ref{table01}处的例子}
\label{call01}
\begin{minted}{sql}
declare 
	c_id customers.id%type := 1;
	c_sal customers.salary%type;
begin
	select salary
	into c_sal
	from customers
	where id = c_id;
	if (c_sal <= 2000) then
		update customers
		set salary = salary + 1000
		where id = c_id;
		dbms_output.put_line('Salary updated');
	end if;
end;
/
\end{minted}

\subsubsection{\code{if-then-elsif-else-end if}}
\begin{minted}{sql}
declare
	a number(3) := 100;
begin
	if (a = 10) then
		dbms_output.put_line('Value of a is 10');
	elsif (a = 20) then
		dbms_output.put_line('Value of a is 20');
	elsif (a = 30) then
		dbms_output.put_line('Value of a is 30');
	else
		dbms_output.put_line('None of the values is matching');
	end if;
	dbms_output.put_line('Exact value of a is: ' || a);
end;
/
\end{minted}

%\subsection{}
%\begin{minted}{sql}
%\end{minted}
\section{附录}

\section{尾注}
\subsection{参考}
\begin{itemize}
\item[(1)] 本文的所有例子基本来自网络,如有侵权请联系本人,本文档按照GPL协议发布。
\item[(2)] 参考网站:\url{http://tutorialspoint.com}
\end{itemize}
\end{document}


