\documentclass{article}
\let\tldocenglish=1  % for live4ht.cfg
\usepackage{tex-live-zh-cn, indentfirst}
\usepackage{graphicx}
\usepackage{color}
\usepackage{xcolor}
\usepackage{verbatim}
\usepackage{comment}
\usepackage{minted}
\definecolor{bg}{rgb}{0.95,0.95,0.95}

\begin{document}

\title{%
  {\huge \textsf{Linux 动态链接库}\\\smallskip}%
  {\small \textit{关于动态链接库的使用技巧}}
}

\author{\textit{屈庆磊}\\[2mm]
        \email{quqinglei@icloud.com} 
       }

\date{\textit{2013 年 8 月 14 日}}

\maketitle
\begin{multicols}{2}
\tableofcontents
\end{multicols}

\section{动态链接库百科}
动态链接库文件,是一种不可执行的二进制程序文件,它允许程序共享执行特殊任务所必需的代码和其他资源。
Windows提供的DLL文件中包含了允许基于Windows的程序在Windows环境下操作的许多函数和资源。一般被存放
在C:\dirname{/Windows/System}目录下。Windows中,DLL多数情况下是带有DLL扩展名的文件,但也可能是EXE
或其他扩展名;Debian系统中常常是.so的文件。它们向运行于Windows操作系统下的程序提供代码、数据或函数。
程序可根据DLL文件中的指令打开、启用、查询、禁用和关闭驱动程序。

\section{动态链接库搜索路径}
\textit{
在Linux中,动态库的搜索路径除了默认的搜索路径外,还可以通过以下三种方式来指定,这个在实现上并没有
什么难度,只是一种规定而已。下面讲述三种方式。
}

\subsection{示例代码}
\begin{minted}{c}
// dym.c
#include <stdio.h>
void test()
{
	printf("Hello world!\n");
}

// main.c
#include <stdio.h>
void test();
int main()
{
	test();
	return 0;
}

// Makefile
all: a.out

a.out: main.o libdym.so
	gcc -o a.out main.o -L. -ldym

# 一定要命名为libxxx.so 否则在使用 -L. -ldym 时会找不到
# 因为默认情况下动态链接库的开头是lib
libdym.so: dym.o
	gcc -shared -fPIC -o libdym.so dym.o

main.o: main.c
	gcc -c main.c

dym.o: dym.c
	gcc -c dym.c

clean: 
	rm -f *.o *.so a.out
\end{minted}

按照上面的代码和makefile 编译后生成a.out和libdym.so 文件此时运行a.out 必然会有如下提示:

\textit{
\dirname{./a.out}: error while loading shared libraries: libdym.so: cannot open shared object file: 
No such file or directory
}

\subsection{全局法}
\textit{
在上一小节中我们编译了一个程序叫:a.out,但无法执行,原因很简单,就是找不到动态链接库,使用ldd\footnote{
ldd 是用来打印程序对动态链接库依赖的小工具,你可以使用man ldd 查看手册。
} a.out
可以看到如下提示:
}

\begin{minted}{sh}
	linux-gate.so.1 =>  (0xb77c5000)
	libdym.so => not found # 找不到此文件
	libc.so.6 => /lib/i386-linux-gnu/i686/cmov/libc.so.6 (0xb7655000)
	/lib/ld-linux.so.2 (0xb77c6000)
\end{minted}

我们把动态链接库libdym.so 的路径导出来写到:\dirname{/etc/ld.so.conf} 里,比如我们现在的路径是:
\dirname{/home/test/hello/} 那么可以在\dirname{/etc/ld.so.conf} 文件里写上如下所示:
\begin{minted}{sh}
include /etc/ld.so.conf.d/*.conf # 默认的,我们也可以写到/etc/ld.so.conf.d/hello.conf
# hello.conf 是我随便起的名字,看上一行有一个include,它会把/etc/ld.so.conf.d/路径里的
# 所有.conf文件包含,所以写到此路径命名为xxx.conf的文件也可以达到同样的目的
/home/test/hello/
\end{minted}

写完以后在root用户下执行ldconfig,此时我们可以执行: ldd a.out,我们可以得到正常的显示结果:
\begin{minted}{sh}
	linux-gate.so.1 =>  (0xb7703000)
	libdym.so => /home/git/papers/dym/aaaa/libdym.so (0xb76f4000)
	libc.so.6 => /lib/i386-linux-gnu/i686/cmov/libc.so.6 (0xb7591000)
	/lib/ld-linux.so.2 (0xb7704000)
\end{minted}

运行a.out 会得到正常的显示结果,如下:

\begin{minted}{sh}
root@debian:/home/test# ./a.out 
Hello world!
\end{minted}

\subsection{导出环境变量法}
\textit{
有些时候你作为普通用户没有root的权限,在全局法使用上收到限制,此时你可以考虑使用导出全局变量法,如下所示:
}

\begin{minted}{sh}
root@debian:/home/test# export LD_LIBRARY_PATH="/lib:/user/lib:/home/test/hello"
root@debian:/home/test# ./a.out 
Hello world!
\end{minted}

你完全可以把环境变量写到一个脚本中,把可执行程序也写到脚本里执行,在导出环境变量后执行即可!如下:

\begin{minted}{sh}
#!/bin/bash
export LD_LIBRARY_PATH="/usr/lib:/lib:/home/test/hello"
./a.out
\end{minted}

\subsection{编译时确定程序的动态库搜索路径}
\textit{
这种方法是可行的,但我不建议这样玩,环境一变就乱套了!不过还是讲一下吧。
}

\begin{minted}{sh}
// 更改后的 Makefile
all: a.out

a.out: main.o libdym.so
	gcc -o a.out main.o -L. -ldym -Wl,-rpath,./

libdym.so: dym.o
	gcc -shared -fPIC -o libdym.so dym.o

main.o: main.c
	gcc -c main.c

dym.o: dym.c
	gcc -c dym.c

clean: 
	rm -f *.o *.so a.out

\end{minted}

以上例子可以直接执行,不需要再添加环境变量或者更改系统配置文件了

\section{程序对动态链接库搜索的顺序}
\textit{
我们可以完全自己写代码测出来程序的寻找先后顺序,比如你在每一个路径都写一个不同输出内容的同名函数,然后
编译成动态链接库,执行代码测试。关于此例我就不多言了。
}

\begin{itemize}
\item[(1)] 最先搜索的是程序编译时指定的动态链接库路径
\item[(2)] 然后是环境变量LD\_LIBRARY\_PATH 指定的搜索路径
\item[(3)] 然后是ld.so.conf 文件包含的路径
\item[(4)] 如果前三个路径都没有,程序会去\dirname{/lib} 路径查找
\item[(5)] 最后会去\dirname{/usr/lib}路径去查找
\end{itemize}

\end{document}

