\documentclass{article}
\let\tldocenglish=1  % for live4ht.cfg

\usepackage{graphicx}
\usepackage{color}
\usepackage{xcolor}
\usepackage{verbatim}
\usepackage{comment}
\usepackage{minted}
\definecolor{bg}{rgb}{0.95,0.95,0.95}
\usepackage{tex-live-zh-cn, indentfirst}

\begin{document}

\title{%
  {\huge \textit{Data Audit System Design}\\\smallskip}%
  {\LARGE \textsf{JSON API PART}}
}

\author{Qu Qinglei 编写 \\[3mm]
        \email{quqinglei@icloud.com}
       }

\date{2013 年 5 月}

\maketitle

\begin{multicols}{2}
\tableofcontents
%\listoftables
\end{multicols}

%[A]
\section{插件管理}
\subsection{安装插件的命令}
\begin{minted}{js}
{
    "type": "addPlugin",
    "plugType": "parser",
}
\end{minted}
\textsf{安装插件的过程}
\begin{itemize}
\item[(1)] 在WEBUI端先上传插件,上传至\dirname{/tmp/plugin/}
\item[(2)] 执行安装插件命令,调用python安装脚本,python程序先解包后释放到\dirname{/opt/crouse/lib/plugin}
\item[(3)] 在数据库中添加这个插件的路径、名称、类型等
\item[(4)] python处理完以上工作后,发送命令给后台守护进程,后台守护进程会根据要求安装到内存
\end{itemize}

\subsection{插件配置}
\begin{minted}{js}
{
    "type": "setPluginState",
    "plugName": "messageParser",
    "status": false
}
\end{minted}
\textsf{插件是否启用}\footnote{默认的情况下插件在安装后会直接启用,此功能提供使插件启用或者不启用}
\begin{itemize}
\item[(1)] WEBUI 可以从数据库中读取所有插件,并提供搜索功能,用户选择插件
\item[(2)] 用户对插件设置启用或者不启用后,点击提交表单,调用插件配置功能
\item[(3)] 执行python脚本根据提交的表单发送命令到后台守护进程,守护进程根据配置要求执行从内存中去掉插件或者
把插件插入到内存
\item[(4)] 执行完以上三步后更改数据库状态
\end{itemize}

\subsection{删除插件的命令}
\begin{minted}{js}
{
    "type": "removePlugin",
    "plugType": "parser",
    "plugName": "messagesParser",
    "dropDatabase": true
}
\end{minted}
\textsf{删除插件的过程}
\begin{itemize}
\item[(1)] 用户在WEBUI上找到删除插件功能,浏览插件
\item[(2)] 用户选择要删除的插件,然后按确定提交表单
\item[(3)] python接收到删除插件的命令后会通知后台守护进程
\item[(4)] 后台守护进程会从内存中删掉插件,并从硬盘中删除该插件文件,并设置插件表为已删除
\end{itemize}


%[B]
\section{日志管理}
\subsection{添加监听日志的命令}
\begin{minted}{js}
{
    "type": "addDatatype",
    "dataType": "messages",
    "port": 514,
    "description": "about this kind of message, can be null."
}
\end{minted}

\subsection{删除监听日志的命令}
\begin{minted}{js}
{
    "type": "removeDatatype",
    "dataType": "messages",
}
\end{minted}

%[C]
\section{报警管理}
\subsection{添加报警条件的命令}
\begin{minted}{js}
{
    "type": "addAlarm",
    "alarmName": "Intrusion Detection",
    "dataType": "messages",
    "globalAlarm": false,
    "alarms": {
        "startTime": "2013-10-11:20:30:15",
        "endTime": "2013-10-11:20:30:15",
        "keywords": "attack#segment falt#stackoverflow#Job `cron.daily' terminated"
    }
}
\end{minted}

\subsection{删除报警条件的命令}
\begin{minted}{js}
{
    "type": "removeAlarm",
    "dataType": "messages",
    "alarmName": "Intrusion Detection"
}
\end{minted}

%[D]
\section{过滤器管理}
\subsection{添加过滤器}
\begin{minted}{js}
{
    "type": "addFilter",
    "filterName": "useless",
    "dataType": "messages",
    "globalFilter": false,
    "filters": {
        "startTime": "2013-10-11:20:30:15",
        "endTime": "2013-10-11:20:30:15",
        "keywords": "useless#(CRON)#DHCPREQUEST"
    }
}
\end{minted}
\subsection{删除过滤器}
\begin{minted}{js}
{
    "type": "removeFilter",
    "filterName": "useless",
    "dataType": "messages"
}
\end{minted}

%[E]
\section{邮箱配置管理}
\subsection{添加邮箱地址}
\begin{minted}{js}
{
    "type": "addEmail",
    "emailName": "quqinglei.aries@gmail.com",
    "mailCategory": "admin"
}
\end{minted}

\subsection{更改邮箱地址}
\begin{minted}{js}
{
    "type": "editEmail",
    "emailName": "quqinglei.aries@gmail.com",
}
\end{minted}

\subsection{删除邮箱地址}
\begin{minted}{js}
{
    "type": "removeEmail",
    "emailName": "quqinglei.aries@gmail.com",
}
\end{minted}

%[F]
\section{数据管理}
\subsection{删除所有本地报警数据}
\begin{minted}{js}
{
    "type": "removeAllalarmData"
}
\end{minted}

%[G]
\section{默认配置管理}
\subsection{设置数据保留的时间}
\begin{minted}{js}
{
    "type": "setDataRetentionTime",
    "last": 365
}
\end{minted}

%[F]
\section{尾注}
本文档主要涉及\code{JSON API}的设计格式,力求最简,在后期可能会有些改动,应该不大。版本
更改内容会在尾注上标注。

Thu May 23 17:05:39 CST 2013 第一次编写



\end{document}



