\documentclass{article}
\let\tldocenglish=1  % for live4ht.cfg
\usepackage{tex-live-zh-cn, indentfirst}
\usepackage{graphicx}
\usepackage{color}
\usepackage{xcolor}
\usepackage{verbatim}
\usepackage{comment}
\usepackage{minted}
\definecolor{bg}{rgb}{0.95,0.95,0.95}

\begin{document}

\title{%
  {\huge \textsf{hadoop configuration}\\\smallskip}%
  {\small \textit{关于hadoop的配置笔记}}
}

\author{\textit{屈庆磊}\\[2mm]
        \email{quqinglei@icloud.com} 
       }

\date{\textit{2013 年 7 月}}

\maketitle
\begin{multicols}{2}
\tableofcontents
%\listoftables
\end{multicols}

\section{配置\code{JAVA}环境}
\subsection{配置\code{JAVA\_HOME}}
\textit{需要根据系统位数下载不同的安装包,如:32位的\code{jdk-7u25-linux-i586.gz},或者64位的\code{jdk-7u25-linux-x64.tar.gz}
建议去甲骨文官网下载。
}

\textsf{例子:}
\begin{minted}{sh}
tar xzv jdk-7u25-linux-i586.gz -C /opt
cd /opt
ln -s jdk1.7.0_25 jdk
echo "export JAVA_HOME=/opt/jdk" >> /etc/profile
source /etc/profile
\end{minted}

\section{\code{Hadoop}安装包}
\textit{由于在Linux下有很多安装方式,为了统一,本次采用解压包介绍。}
\textit{去Hadoop的官网,找到Hadoop的下载地址,在稳定版本里挑一个下载,本例中采用的包为:\code{hadoop-1.1.2.tar.gz}}


\section{Hadoop 集群安装实例}
\textit{本例中有三台机器参与,一台master,两台slave,配置比较简单}

\subsection{说明}
\begin{description}
\item[192.168.1.5] 此台机器作为\code{master}
\item[192.168.1.6] 此台机器作为\code{slave01}
\item[192.168.1.7] 此台机器作为\code{slave02}
\end{description}

\subsection{第二步,配置hosts}
\textit{配置Hosts:}
编辑\code{master}机器的\dirname{/etc/hosts}文件,添加内容为如下所示:

\begin{minted}{sh}
192.168.1.5 master
192.168.1.6 slave01
192.168.1.7 slave02
\end{minted}

同步此配置文件到其它两台机器,保持一致即可。

\subsection{第三步,配置\code{JAVA\_HOME}见上述章节}

\subsection{第四步,建立新用户,并解压安装包}
\begin{minted}{sh}
useradd -m hadoop
passwd hadoop #设置密码
su hadoop
cd /home/hadoop
tar xzf hadoop-1.1.2.tar.gz 
ln -s hadoop-1.1.2 hadoop
#hadoop 的家路径此时为:/home/hadoop/hadoop
\end{minted}

\subsection{第四步,配置\code{ssh}无密码访问}
\textit{其实就是交换公匙}
\begin{minted}{sh}
# machine master, 使用hadoop用户
ssh-keygen -t dsa -P '' -f ~/.ssh/id_dsa
cat ~/.ssh/id_dsa.pub >> ~/.ssh/authorized_keys
scp ~/.ssh/id_dsa.pub hadoop@192.168.1.6:/home/hadoop/.ssh/authorized_keys_master
scp ~/.ssh/id_dsa.pub hadoop@192.168.1.6:/home/hadoop/.ssh/authorized_keys_master

# slave01, 使用hadoop用户
ssh-keygen -t dsa -P '' -f ~/.ssh/id_dsa
cat ~/.ssh/id_dsa.pub >> ~/.ssh/authorized_keys
cat ~/.ssh/authorized_keys_master >> ~/.ssh/authorized_keys
scp ~/.ssh/id_dsa.pub hadoop@192.168.1.5:/home/hadoop/.ssh/authorized_keys_slave01

# slave02, 使用hadoop用户
ssh-keygen -t dsa -P '' -f ~/.ssh/id_dsa
cat ~/.ssh/id_dsa.pub >> ~/.ssh/authorized_keys
cat ~/.ssh/authorized_keys_master >> ~/.ssh/authorized_keys
scp ~/.ssh/id_dsa.pub hadoop@192.168.1.5:/home/hadoop/.ssh/authorized_keys_slave02

# master, 使用hadoop用户
cat ~/.ssh/authorized_keys_slave01 >> ~/.ssh/authorized_keys
cat ~/.ssh/authorized_keys_slave02 >> ~/.ssh/authorized_keys
\end{minted}

\textbf{注意:在CentOS下做的同学需要注意,一定要看看~/.ssh/authorized\_keys的权限是不是644,如果
不是644,请\code{chmod 644 authorized\_keys}}

\subsection{第五步,更改Hadoop配置文件}
\begin{minted}{sh}
su hadoop
cd /home/hadoop/hadoop/conf/
# 需要更改:masters, slaves, hadoop-env.sh, core-site.xml, hdfs-site.xml, mapred-site.xml
\end{minted}

\textit{更改结果如下所示:}

\begin{minted}{sh}
#masters 文件内容
master

#slaves 文件内容
slave01
slave02

#hadoop-env.sh
#仅仅需要更改JAVA_HOME那一行,更改为JAVA_HOME的实际路径,本例中我们采用的JAVA_HOME为:
export JAVA_HOME=/opt/jdk
\end{minted}

\begin{minted}{xml}
#core-site.xml 文件内容
<?xml version="1.0"?>
<?xml-stylesheet type="text/xsl" href="configuration.xsl"?>

<!-- Put site-specific property overrides in this file. -->

<configuration>
<property>
<name>fs.default.name</name>
<value>hdfs://master:9000</value>
</property>
</configuration>

#hdfs-site.xml 文件内容
<?xml version="1.0"?>
<?xml-stylesheet type="text/xsl" href="configuration.xsl"?>

<!-- Put site-specific property overrides in this file. -->

<configuration>
<property>
<name>dfs.name.dir</name>
<value>/home/hadoop/hadoopfs/name1,/home/hadoop/hadoopfs/name2</value>
</property>

<property>
<name>dfs.data.dir</name>
<value>/home/hadoop/hadoopfs/data1,/home/hadoop/hadoopfs/data2</value>
</property>

<property>
<name>dfs.replication</name>
<value>2</value>
</property>
</configuration>

#mapred-site.xml 文件内容
<?xml version="1.0"?>
<?xml-stylesheet type="text/xsl" href="configuration.xsl"?>

<!-- Put site-specific property overrides in this file. -->

<configuration>
<property>
<name>mapred.job.tracker</name>
<value>master:9200</value>
</property>
</configuration>
\end{minted}

\subsection{第六步,把hadoop打包,复制到其他两台机器}
\textit{在配置完一台机器后,直接可以把配置好的hadoop复制到其他机器上,当然在其他的
机器上的位置应该是一样的}

\subsection{完毕}
基本配置完毕,在主机的\dirname{/home/hadoop/hadoop/bin}下执行:
\code{./hadoop fs namenode -format} 格式化文件系统,然后就可以启动hadoop集群了,
\code{./start-all.sh}即可启动hadoop集群

\begin{minted}{sh}
cd /home/hadoop/hadoop
./hadoop fs namenode -format
./start-all.sh
\end{minted}

如果成功,我们可以在其他机器上都能看到hadoop的进程,并且在\dirname{/home/hadoop/}会看到\dirname{/home/hadoop/hadoopfs}的路径
出现。

\section{编译Hadoop的FUSE模块,机器一定要能上网}
\textit{如果想在其他机器上挂载HDFS,则需要编译 fuse\_dfs 模块}
\subsection{编译前的软件依赖安装}
\begin{minted}{sh}
# 如果系统是基于Redhat的,请使用yum安装,如果基于debian,使用apt-get 安装,具体如下:
yum install install automake autoconf m4 libtool pkgconfig fuse fuse-devel fuse-libs

apt-get install automake autoconf m4 libtool libextutils-pkgconfig-perl \
	 fuse libfuse-dev lrzsz build-essential
\end{minted}

\subsection{下载ant包,编译需要它}
\begin{minted}{sh}
tar xzf apache-ant-1.9.1-bin.tar.gz -C /opt
cd /opt
ln -s apache-ant-1.9.1 ant

\end{minted}

\subsection{设置环境变量}
\textit{把下面环境变量写到\dirname{/etc/profile}文件里}
\begin{minted}{sh}
# add below to /etc/profile
export JAVA_HOME=/opt/jdk
export HADOOP_HOME=/home/hadoop/hadoop
export OS_ARCH=i386 #or amd64
export OS_BIT=32 #or 64
export ANT_HOME=/opt/ant
export PATH=$PATH:$ANT_HOME/bin
export LD_LIBRARY_PATH=$JAVA_HOME/jre/lib/$OS_ARCH/server:\ 
	$HADOOP_HOME/build/c++/Linux-$OS_ARCH-$OS_BIT/lib:/usr/local/lib:/usr/lib
\end{minted}

\subsubsection{更新环境变量}
\code{source /etc/profile}

\subsubsection{编译HDFS}
\begin{minted}{sh}
cd /home/hadoop/hadoop # 进入hadoop的家路径
ant compile-c++-libhdfs -Dlibhdfs=1 -Dcompile.c++=1 #编译libhdfs,机器一定要能上网,否则无法编译
\end{minted}

\subsection{编译fuse\_dfs}
\begin{minted}{sh}
# 进入hadoop家路径,执行
ln -s c++/Linux-$OS_ARCH-$OS_BIT/lib build/libhdfs 
ant compile-contrib -Dlibhdfs=1 -Dfusedfs=1 
\end{minted}

\textsf{如果以上不出错,则说明问题解决}
编译完成后挂载就很简单了,但应该注意以下问题:
\begin{itemize}
\item[(1)] 编辑\dirname{/etc/ld.so.conf} 添加libhdfs.so文件的路径,或者可以直接把:\dirname{/home/hadoop/hadoop/c++/Linux-i386-32/lib/} 里
的所有文件拷贝到\dirname{/usr/lib}路径,记得执行\code{ldconfig}
\item[(2)] 可以把\dirname{/home/hadoop/build/contrib/fuse-dfs} 路径里的两个文件拷贝到\dirname{/usr/local/bin/} 但必须更改\dirname{fuse_dfs_wrapper.sh}
的最后一行\code{./fuse\_dfs} 为:\code{/usr/local/bin/fuse\_dfs}, \dirname{fuse_dfs_wrapper.sh} 是个脚本,请给予执行权限,并可以更改里面的环境变量为
真正的值,如果环境变量已设为全局,则不需要更改。
\end{itemize}





\end{document}


























