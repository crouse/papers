\documentclass{article}
\let\tldocenglish=1  % for live4ht.cfg
\usepackage{tex-live-zh-cn, indentfirst}
\usepackage{graphicx}
\usepackage{color}
\usepackage{xcolor}
\usepackage{verbatim}
\usepackage{comment}
\usepackage{minted}
\usepackage{float}

\begin{document}

\title{%
  {\huge \textsf{hadoop config and management}\\\smallskip}%
  {\small \textit{关于hadoop配置管理的手册}}
}

\author{\textit{屈庆磊}\\[2mm]
        \email{quqinglei@icloud.com} 
       }

\date{\textit{2013 年 7 月}}

\maketitle
\begin{multicols}{2}
\tableofcontents
%\listoftables
\end{multicols}

\section{写在前面}
此文档或许有存在的意义吧,如果觉得不爽,删掉就OK啦,文档目的是注重实际,本来是我的学习笔记,未来希望能写成
一本很容易理解又很容易上手的使用手册,面向一些具体的问题进行分析、解决,后期将会从Java API 的角度,一些特定
mapreduce 案例的编写 来讲解hadoop,如果一切顺利,此文档在后面将对Hadoop本身代码进行分析,可此路荆棘,非一时
之功,后期将讲解如何优化 Hadoop,并给出一些测试方法和衡量方法。内容大部分来自网络的各位网友们的总结以及Hadoop
官网给出的帮助文档。本文档将尽可能的短小精悍,未来将控制在200页左右,并由易到难,而不是和书本式的照本宣科。
此文本身可能出现一些写作上的失误或者错误,如有发现请邮件我,谢谢!

\section{环境变量}
本例中采用和很多环境变量,如:\code{JAVA\_HOME HADOOP\_HOME} 等等,建议在做以下工作的时候,先找一个文本文件,把环境变量都写好,
然后在写入\dirname{/etc/profile} 文件,当然有些环境变量只是在编译的时候使用,这个你可以根据情况酌情处理,当然环境变量都导出来
也是没有什么问题的。

\section{配置\code{JAVA}环境}
\subsection{配置\code{JAVA\_HOME}}
\textit{需要根据系统位数下载不同的安装包,如:32位的\code{jdk-7u25-linux-i586.gz},或者64位的\code{jdk-7u25-linux-x64.tar.gz}
建议去甲骨文官网下载。
}

\textsf{例子:}
\begin{minted}{sh}
tar xzv jdk-7u25-linux-i586.gz -C /opt
cd /opt
ln -s jdk1.7.0_25 jdk
echo "export JAVA_HOME=/opt/jdk" >> /etc/profile
source /etc/profile
\end{minted}

\section{\code{Hadoop}安装包}
\textit{由于在Linux下有很多安装方式,为了统一,本次采用解压包介绍。}
\textit{去Hadoop的官网,找到Hadoop的下载地址,在稳定版本里挑一个下载,本例中采用的包为:\code{hadoop-1.1.2.tar.gz}}


\section{Hadoop 集群安装实例}
\textit{本例中有三台机器参与,一台master,两台slave,配置比较简单}

\subsection{说明}
\begin{description}
\item[192.168.1.5] 此台机器作为\code{master}
\item[192.168.1.6] 此台机器作为\code{slave01}
\item[192.168.1.7] 此台机器作为\code{slave02}
\end{description}

\subsection{第二步,配置hosts}
\textit{配置Hosts:}
编辑\code{master}机器的\dirname{/etc/hosts}文件,添加内容为如下所示:

\begin{minted}{sh}
192.168.1.5 master
192.168.1.6 slave01
192.168.1.7 slave02
\end{minted}

同步此配置文件到其它两台机器,保持一致即可。

\subsection{第三步,配置\code{JAVA\_HOME}见上述章节}

\subsection{第四步,建立新用户,并解压安装包}
\begin{minted}{sh}
useradd -m hadoop
passwd hadoop #设置密码
su hadoop
cd /home/hadoop
tar xzf hadoop-1.1.2.tar.gz 
ln -s hadoop-1.1.2 hadoop
#hadoop 的家路径此时为:/home/hadoop/hadoop
\end{minted}

\subsection{第四步,配置\code{ssh}无密码访问}
\textit{其实就是交换公匙}
\begin{minted}{sh}
# machine master, 使用hadoop用户
ssh-keygen -t dsa -P '' -f ~/.ssh/id_dsa
cat ~/.ssh/id_dsa.pub >> ~/.ssh/authorized_keys
scp ~/.ssh/id_dsa.pub hadoop@192.168.1.6:/home/hadoop/.ssh/authorized_keys_master
scp ~/.ssh/id_dsa.pub hadoop@192.168.1.6:/home/hadoop/.ssh/authorized_keys_master

# slave01, 使用hadoop用户
ssh-keygen -t dsa -P '' -f ~/.ssh/id_dsa
cat ~/.ssh/id_dsa.pub >> ~/.ssh/authorized_keys
cat ~/.ssh/authorized_keys_master >> ~/.ssh/authorized_keys
scp ~/.ssh/id_dsa.pub hadoop@192.168.1.5:/home/hadoop/.ssh/authorized_keys_slave01

# slave02, 使用hadoop用户
ssh-keygen -t dsa -P '' -f ~/.ssh/id_dsa
cat ~/.ssh/id_dsa.pub >> ~/.ssh/authorized_keys
cat ~/.ssh/authorized_keys_master >> ~/.ssh/authorized_keys
scp ~/.ssh/id_dsa.pub hadoop@192.168.1.5:/home/hadoop/.ssh/authorized_keys_slave02

# master, 使用hadoop用户
cat ~/.ssh/authorized_keys_slave01 >> ~/.ssh/authorized_keys
cat ~/.ssh/authorized_keys_slave02 >> ~/.ssh/authorized_keys
\end{minted}

\textbf{注意:在CentOS下做的同学需要注意,一定要看看~/.ssh/authorized\_keys的权限是不是644,如果
不是644,请\code{chmod 644 authorized\_keys}}

\subsection{第五步,更改Hadoop配置文件}
\begin{minted}{sh}
su hadoop
cd /home/hadoop/hadoop/conf/
# 需要更改:masters, slaves, hadoop-env.sh, core-site.xml, hdfs-site.xml, mapred-site.xml
\end{minted}

\textit{更改结果如下所示:}

\begin{minted}{sh}
#masters 文件内容
master

#slaves 文件内容
slave01
slave02

#hadoop-env.sh
#仅仅需要更改JAVA_HOME那一行,更改为JAVA_HOME的实际路径,本例中我们采用的JAVA_HOME为:
export JAVA_HOME=/opt/jdk
\end{minted}

\begin{minted}{xml}
#core-site.xml 文件内容
<?xml version="1.0"?>
<?xml-stylesheet type="text/xsl" href="configuration.xsl"?>

<!-- Put site-specific property overrides in this file. -->

<configuration>
<property>
<name>fs.default.name</name>
<value>hdfs://master:9000</value>
</property>
</configuration>

#hdfs-site.xml 文件内容
<?xml version="1.0"?>
<?xml-stylesheet type="text/xsl" href="configuration.xsl"?>

<!-- Put site-specific property overrides in this file. -->

<configuration>
<property>
<name>dfs.name.dir</name>
<value>/home/hadoop/hadoopfs/name1,/home/hadoop/hadoopfs/name2</value>
</property>

<property>
<name>dfs.data.dir</name>
<value>/home/hadoop/hadoopfs/data1,/home/hadoop/hadoopfs/data2</value>
</property>

<property>
<name>dfs.replication</name>
<value>2</value>
</property>
</configuration>

#mapred-site.xml 文件内容
<?xml version="1.0"?>
<?xml-stylesheet type="text/xsl" href="configuration.xsl"?>

<!-- Put site-specific property overrides in this file. -->

<configuration>
<property>
<name>mapred.job.tracker</name>
<value>master:9200</value>
</property>
</configuration>
\end{minted}

\subsection{第六步,把hadoop打包,复制到其他两台机器}
\textit{在配置完一台机器后,直接可以把配置好的hadoop复制到其他机器上,当然在其他的
机器上的位置应该是一样的}

\subsection{完毕}
基本配置完毕,在主机的\dirname{/home/hadoop/hadoop/bin}下执行:
\code{./hadoop namenode -format} 格式化文件系统,然后就可以启动hadoop集群了,
\code{./start-all.sh}即可启动hadoop集群

\begin{minted}{sh}
cd /home/hadoop/hadoop
./hadoop fs namenode -format
./start-all.sh
\end{minted}

如果成功,我们可以在其他机器上都能看到hadoop的进程,并且在\dirname{/home/hadoop/}会看到\dirname{/home/hadoop/hadoopfs}的路径
出现。

\section{编译Hadoop的FUSE模块,机器一定要能上网}
\textit{如果想在其他机器上挂载HDFS,则需要编译 fuse\_dfs 模块}
\subsection{编译前的软件依赖安装}
\begin{minted}{sh}
# 如果系统是基于Redhat的,请使用yum安装,如果基于debian,使用apt-get 安装,具体如下:
yum install install automake autoconf m4 libtool pkgconfig fuse fuse-devel fuse-libs

apt-get install automake autoconf m4 libtool libextutils-pkgconfig-perl \
	 fuse libfuse-dev lrzsz build-essential
\end{minted}

\subsection{下载ant包,编译需要它}
\begin{minted}{sh}
tar xzf apache-ant-1.9.1-bin.tar.gz -C /opt
cd /opt
ln -s apache-ant-1.9.1 ant

\end{minted}

\subsection{设置环境变量}
\textit{把下面环境变量写到\dirname{/etc/profile}文件里}
\begin{minted}{sh}
# add below to /etc/profile
export JAVA_HOME=/opt/jdk
export HADOOP_HOME=/home/hadoop/hadoop
export OS_ARCH=i386 #or amd64
export OS_BIT=32 #or 64
export ANT_HOME=/opt/ant
export PATH=$PATH:$ANT_HOME/bin
export LD_LIBRARY_PATH=$JAVA_HOME/jre/lib/$OS_ARCH/server: \
	$HADOOP_HOME/build/c++/Linux-$OS_ARCH-$OS_BIT/lib:/usr/local/lib:/usr/lib
\end{minted}

\subsubsection{更新环境变量}
\code{source /etc/profile}

\subsubsection{编译HDFS}
\begin{minted}{sh}
cd /home/hadoop/hadoop # 进入hadoop的家路径
ant compile-c++-libhdfs -Dlibhdfs=1 -Dcompile.c++=1 #编译libhdfs,机器一定要能上网,否则无法编译
\end{minted}

\subsection{编译fuse\_dfs}
\begin{minted}{sh}
# 进入hadoop家路径,执行
ln -s c++/Linux-$OS_ARCH-$OS_BIT/lib build/libhdfs 
ant compile-contrib -Dlibhdfs=1 -Dfusedfs=1 
\end{minted}

\textsf{如果以上不出错,则说明问题解决}
编译完成后挂载就很简单了,但应该注意以下问题:
\begin{itemize}
\item[(1)] 编辑\dirname{/etc/ld.so.conf} 添加libhdfs.so文件的路径,或者可以直接把:\dirname{/home/hadoop/hadoop/c++/Linux-i386-32/lib/} 里
的所有文件拷贝到\dirname{/usr/lib}路径,记得执行\code{ldconfig}
\item[(2)] 可以把\dirname{/home/hadoop/build/contrib/fuse-dfs} 路径里的两个文件拷贝到\dirname{/usr/local/bin/} 但必须更改\dirname{fuse_dfs_wrapper.sh}
的最后一行\code{./fuse\_dfs} 为:\code{/usr/local/bin/fuse\_dfs}, \dirname{fuse_dfs_wrapper.sh} 是个脚本,请给予执行权限,并可以更改里面的环境变量为
真正的值,如果环境变量已设为全局,则不需要更改。
\end{itemize}


\section{编译并执行经典的例子程序WordCount}
\subsection{例子代码}
\textit{此例子是经过改动的,如果直接使用Hadoop官网的例子在新版本的Hadoop 1.1.2会出错,改动处有标记。}
\begin{minted}{java}
// WordCount.java
package org.myorg;

import java.io.IOException;
import java.util.*;

import org.apache.hadoop.fs.Path;
import org.apache.hadoop.conf.*;
import org.apache.hadoop.io.*;
import org.apache.hadoop.mapreduce.*;
import org.apache.hadoop.mapreduce.lib.input.FileInputFormat;
import org.apache.hadoop.mapreduce.lib.input.TextInputFormat;
import org.apache.hadoop.mapreduce.lib.output.FileOutputFormat;
import org.apache.hadoop.mapreduce.lib.output.TextOutputFormat;

public class WordCount {

	public static class Map extends Mapper<LongWritable, Text, Text, IntWritable> {
		private final static IntWritable one = new IntWritable(1);
		private Text word = new Text();

		public void map(LongWritable key, Text value, Context context) 
			throws IOException, InterruptedException {
				String line = value.toString();
				StringTokenizer tokenizer = new StringTokenizer(line);
				while (tokenizer.hasMoreTokens()) {
					word.set(tokenizer.nextToken());
					context.write(word, one);
				}
			}
	} 

	public static class Reduce extends Reducer<Text, IntWritable, Text, IntWritable> {

		public void reduce(Text key, Iterable<IntWritable> values, Context context) 
			throws IOException, InterruptedException {
				int sum = 0;
				for (IntWritable val : values) {
					sum += val.get();
				}
				context.write(key, new IntWritable(sum));
			}
	}

	public static void main(String[] args) throws Exception {
		Configuration conf = new Configuration();

		Job job = new Job(conf, "wordcount");

		job.setOutputKeyClass(Text.class);
		job.setOutputValueClass(IntWritable.class);
		job.setJarByClass(WordCount.class); // 多加了这一行!!!

		job.setMapperClass(Map.class);
		job.setReducerClass(Reduce.class);

		job.setInputFormatClass(TextInputFormat.class);
		job.setOutputFormatClass(TextOutputFormat.class);

		FileInputFormat.addInputPath(job, new Path(args[0]));
		FileOutputFormat.setOutputPath(job, new Path(args[1]));

		job.waitForCompletion(true);
	}

}
\end{minted}

\subsection{编译过程}
\textit{废话少说,直接看Makefile和文件列表,我觉得模仿是最好的学习方式}

\begin{minted}{sh}
hadoop@debian:~/testHadoop/wordcount$ tree 
.
├── Makefile #用来编译例子的 Makefile
├── run.sh #用来执行的,其实里面就是一行命令,我懒得记
├── wordcount_classes #这是个空的文件夹,用于放编译完的CLASS文件
└── WordCount.java #这个是上面那个例子的源文件

1 directory, 3 files
\end{minted}

\begin{minted}{sh}
# Makefile 内容
all: wordcount jarpackage

wordcount: 
	javac -classpath ${CLASSPATH} -d  wordcount_classes WordCount.java

jarpackage:
	jar -cvf wordcount.jar -C wordcount_classes/ . 

clean:
	rm -rf wordcount_classes/*
	rm -f wordcount.jar

# run.sh 内容
#!/bin/bash

hadoop fs -rmr /user/hadoop/output/ # output 路径不能存在,否则执行失败
hadoop jar wordcount.jar org.myorg.WordCount input output 
# input output 默认是在/user/hadoop/input /user/hadoop/output
# 如果input路径不是这个请写绝对路径

# 编译后的wordcount_classes 路径:
hadoop@debian:~/testHadoop/wordcount$ tree wordcount_classes/
wordcount_classes/
└── org
    └── myorg
        ├── WordCount.class
        ├── WordCount$Map.class
        └── WordCount$Reduce.class

2 directories, 3 files
\end{minted}

\textsf{注意:执行run.sh时,注意org.myorg.WordCount是否和上面的wordcount\_classes 结构一致!}

\section{hadoop 配置进阶}
\subsection{配置垃圾箱}
\textit{
想必大家都犯过类似的错误,不小心把一大堆数据全删掉了,咋办?这个时候你会很崩溃的,如果是在操作系统的文件系统下
你还能使用恢复软件,检查inode节点,恢复数据,但在hadoop上貌似没那么简单,所以有一部时间机器是很有用的,那么我们
可以开启垃圾箱功能,并设置系统自动删除过期数据的超时时间,这样至少在我们想起删错东西的时候可以挽回带来的损失。
}

\subsubsection{配置方法}
\textit{
只需要更改配置文件:core-site.xml,value的单位是分钟。
}

\begin{minted}{xml}
<property>
	<name>fs.trash.interval</name>
	<value>1440</value>
</property>
\end{minted}

\subsubsection{还原方法}
\textit{
只需要把文件从.Trash文件夹里移动到原来的位置即可
}

\begin{minted}{sh}
# 比如我在/user/test/lei/ 路径删除一个文件 abc,那么我应该去哪找呢?
# 其实位置应该在 /user/hadoop/.Trash/Current/user/test/lei/ 路径去找
# 注意上面路径中的hadoop其实是用户名,我们直接可以把文件移动过去就可以了

$ hadoop fs -mv /user/hadoop/.Trash/Current/user/test/lei/abc /user/test/lei/abc

# 清理垃圾
$ hadoop fs -rmr /user/hadoop/.Trash
$
\end{minted}

\subsection{磁盘预留剩余空间}
\textit{
这个参数有点脱了裤子放屁,对于一个正常的软件都应该有防止硬盘写满的测试,或许是为了速度吧,
在Linux下如果硬盘写满了,会导致一些问题,比如无法写入问题,然后硬盘看着像只读似的,所以尽量
不要让你的硬盘写满,所以这个选项还是有必要的,特别是对于那些硬盘本来就不是多宽裕个兄弟们。
}

下面是配置代码,直接写到: hdfs-site.xml 文件里就可以啦!value的单位是字节

\begin{minted}{xml}
<property> 
<name>dfs.datanode.du.reserved</name> 
<value>10737418240</value> 
<description>Reserved space in bytes per volume
</description> 
</property>
\end{minted}


\section{Hadoop 管理}
\subsection{文件操作类命令}
\textit{
这行命令和一些UNIX命令类似,所以比较容易记住,本文档为学习而写,为记忆而写。
}

\begin{description}
\item[cat] 查看文件内容,类似于UNIX系统中的cat命令

\begin{minted}{sh}
# Usage: hadoop fs -cat URL [URI ...]
hadoop fs -cat hdfs://nn1.example.com/file1 hdfs://nn2.example.com/file2
hadoop fs -cat file://file3 /user/hadoop/file4

#返回值:0: 正确 -1: 错误
\end{minted}


\item[chgrp] 改变文件所属于的组,使用-R参数可以递归执行此路径的所有文件,文件必须属于执行此命令者
或者执行命令者为超级用户。

\begin{minted}{sh}
# Usage: hadoop fs -chgrp [-R] GROUP URI [URI ...]

\end{minted}

%%%%%%%%%%%%%%%%%%%%%%%%%%%%%%%%%%%%%%%%%%%%%%%%%%%%%%%%%%%%%%%%%%%%%%%
\item[chmod] 更改文件的权限,忘UNIX的chmod上靠就对了,使用-R递归整个路径

\begin{minted}{sh}
# Usage: hadoop fs -chmod [-R] [OWNER] [:[GROUP]] URI [URI]
\end{minted}
%%%%%%%%%%%%%%%%%%%%%%%%%%%%%%%%%%%%%%%%%%%%%%%%%%%%%%%%%%%%%%%%%%%%%%%

%%%%%%%%%%%%%%%%%%%%%%%%%%%%%%%%%%%%%%%%%%%%%%%%%%%%%%%%%%%%%%%%%%%%%%%
\item[chown] 改变文件所属于的用户,使用-R递归整个路径

\begin{minted}{sh}
# Usage: hadoop fs -chown [-R] [OWNER] [:[GROUP]] URI [URI]
\end{minted}
%%%%%%%%%%%%%%%%%%%%%%%%%%%%%%%%%%%%%%%%%%%%%%%%%%%%%%%%%%%%%%%%%%%%%%%

%%%%%%%%%%%%%%%%%%%%%%%%%%%%%%%%%%%%%%%%%%%%%%%%%%%%%%%%%%%%%%%%%%%%%%%
\item[copyFromLocal] 和put命令类似,只不过源被限制为本地文件

\begin{minted}{sh}
# Usage: hadoop fs -copyFromLocal <localsrc> URI
\end{minted}
%%%%%%%%%%%%%%%%%%%%%%%%%%%%%%%%%%%%%%%%%%%%%%%%%%%%%%%%%%%%%%%%%%%%%%%

%%%%%%%%%%%%%%%%%%%%%%%%%%%%%%%%%%%%%%%%%%%%%%%%%%%%%%%%%%%%%%%%%%%%%%%
\item[cp] 复制命令

\begin{minted}{sh}
# Usage: hadoop fs -cp URI [URI ...] <dest>
Example: 
hadoop fs -cp /user/hadoop/file1 /user/hadoop/file2
hadoop fs -cp /user/hadoop/file1 /user/hadoop/file2 /user/hadoop/dir

#返回值:0: 正确 -1: 错误
\end{minted}
%%%%%%%%%%%%%%%%%%%%%%%%%%%%%%%%%%%%%%%%%%%%%%%%%%%%%%%%%%%%%%%%%%%%%%%

%%%%%%%%%%%%%%%%%%%%%%%%%%%%%%%%%%%%%%%%%%%%%%%%%%%%%%%%%%%%%%%%%%%%%%%
\item[du] 估计文件空间使用情况

\begin{minted}{sh}
# Usage: hadoop fs -du URI [URI ...]
Example: 
hadoop fs -du /user/hadoop/dir1 /user/hadoop/file2 \
	hdfs://nn.example.com/user/hadoop/dir1


#返回值:0: 正确 -1: 错误
\end{minted}
%%%%%%%%%%%%%%%%%%%%%%%%%%%%%%%%%%%%%%%%%%%%%%%%%%%%%%%%%%%%%%%%%%%%%%%

%%%%%%%%%%%%%%%%%%%%%%%%%%%%%%%%%%%%%%%%%%%%%%%%%%%%%%%%%%%%%%%%%%%%%%%
\item[dus] 估计某路径的空间使用情况

\begin{minted}{sh}
# Usage: hadoop fs -dus <args>
# Example: 
hadoop fs -dus /user/hadoop/dir1 
\end{minted}
%%%%%%%%%%%%%%%%%%%%%%%%%%%%%%%%%%%%%%%%%%%%%%%%%%%%%%%%%%%%%%%%%%%%%%%

%%%%%%%%%%%%%%%%%%%%%%%%%%%%%%%%%%%%%%%%%%%%%%%%%%%%%%%%%%%%%%%%%%%%%%%
\item[expunge] 清空垃圾箱,或者比喻为清空回收站

\begin{minted}{sh}
# Usage: hadoop fs -expunge
\end{minted}
%%%%%%%%%%%%%%%%%%%%%%%%%%%%%%%%%%%%%%%%%%%%%%%%%%%%%%%%%%%%%%%%%%%%%%%

%%%%%%%%%%%%%%%%%%%%%%%%%%%%%%%%%%%%%%%%%%%%%%%%%%%%%%%%%%%%%%%%%%%%%%%
\item[get] 下载文件到本地

\begin{minted}{sh}
# Usage: hadoop fs -get [-ignorecrc] [-crc] <src> <localdst>
# Example: 
hadoop fs -get /user/hadoop/file localfile
hadoop fs -get hdfs://nn.example.com/user/hadoop/file localfile
\end{minted}
%%%%%%%%%%%%%%%%%%%%%%%%%%%%%%%%%%%%%%%%%%%%%%%%%%%%%%%%%%%%%%%%%%%%%%%

%%%%%%%%%%%%%%%%%%%%%%%%%%%%%%%%%%%%%%%%%%%%%%%%%%%%%%%%%%%%%%%%%%%%%%%
\item[getmerge] 下载之前合并,下载后即为合并的文件,也就是说如果在HDFS
上有一个路径,里面都是小文件,我们可以使用此命令下载此路径的所有文件
并在下载后形成一个合并的文件。[addnl] 就是在合并后的文件结尾加个换行符

\begin{minted}{sh}
# Usage: hadoop fs -getmerge <src> <localdst> [addnl]
# Example: 
hadoop fs -getmerge /user/hadoop/test localfile
\end{minted}
%%%%%%%%%%%%%%%%%%%%%%%%%%%%%%%%%%%%%%%%%%%%%%%%%%%%%%%%%%%%%%%%%%%%%%%

%%%%%%%%%%%%%%%%%%%%%%%%%%%%%%%%%%%%%%%%%%%%%%%%%%%%%%%%%%%%%%%%%%%%%%%
\item[ls] 莫要多讲,和UNIX下的ls功能一样

\begin{minted}{sh}
# Usage: hadoop fs -ls <args>
# Example: 
hadoop fs -ls /user/hadoop/test hdfs://nn.example.com/user/hadoop/dir1

#返回值:0: 正确 -1: 错误
\end{minted}
%%%%%%%%%%%%%%%%%%%%%%%%%%%%%%%%%%%%%%%%%%%%%%%%%%%%%%%%%%%%%%%%%%%%%%%

%%%%%%%%%%%%%%%%%%%%%%%%%%%%%%%%%%%%%%%%%%%%%%%%%%%%%%%%%%%%%%%%%%%%%%%

\item[lsr] UNIX下的ls -R功能一样,递归列出本路径所有文件

\begin{minted}{sh}
# Usage: hadoop fs -lsr <args>
\end{minted}
%%%%%%%%%%%%%%%%%%%%%%%%%%%%%%%%%%%%%%%%%%%%%%%%%%%%%%%%%%%%%%%%%%%%%%%

%%%%%%%%%%%%%%%%%%%%%%%%%%%%%%%%%%%%%%%%%%%%%%%%%%%%%%%%%%%%%%%%%%%%%%%
\item[mkdir] UNIX下的mkdir功能一样,创建路径

\begin{minted}{sh}
# Usage: hadoop fs -mkdir <paths>
# Example:
hadoop fs -mkdir /user/hadoop/dir1 /user/hadoop/dir2
hadoop fs -mkdir hdfs://nn1.example.com/user/hadoop/dir \
	 hdfs://nn2.example.com/user/hadoop/dir2

#返回值:0: 正确 -1: 错误
\end{minted}
%%%%%%%%%%%%%%%%%%%%%%%%%%%%%%%%%%%%%%%%%%%%%%%%%%%%%%%%%%%%%%%%%%%%%%%

%%%%%%%%%%%%%%%%%%%%%%%%%%%%%%%%%%%%%%%%%%%%%%%%%%%%%%%%%%%%%%%%%%%%%%%
\item[mv] UNIX下的mv功能一样,移动文件

\begin{minted}{sh}
# Usage: hadoop fs -mv URI [URI ...] <dest>
# Example:
hadoop fs -mv /user/hadoop/file1 /user/hadoop/file2
hadoop fs -mv hdfs://nn.example.com/file1 hdfs://nn.example.com/file2 \
	hdfs://nn.example.com/file3 hdfs://nn.example.com/dir1

#返回值:0: 正确 -1: 错误
\end{minted}
%%%%%%%%%%%%%%%%%%%%%%%%%%%%%%%%%%%%%%%%%%%%%%%%%%%%%%%%%%%%%%%%%%%%%%%

%%%%%%%%%%%%%%%%%%%%%%%%%%%%%%%%%%%%%%%%%%%%%%%%%%%%%%%%%%%%%%%%%%%%%%%
\item[put] 上传文件到HDFS

\begin{minted}{sh}
# Usage: hadoop fs -put <localsrc> ... <dst>
# Example:
hadoop fs -put localfile /user/hadoop/hadoopfile
hadoop fs -put localfile1 localfile2 /user/hadoop/hadoopdir
hadoop fs -put localfile hdfs://nn.example.com/hadoop/hadoopfile
hadoop fs -put - hdfs://nn.example.com/hadoop/hadoopfile # 从标准输入读取,并上传

#返回值:0: 正确 -1: 错误
\end{minted}
%%%%%%%%%%%%%%%%%%%%%%%%%%%%%%%%%%%%%%%%%%%%%%%%%%%%%%%%%%%%%%%%%%%%%%%

%%%%%%%%%%%%%%%%%%%%%%%%%%%%%%%%%%%%%%%%%%%%%%%%%%%%%%%%%%%%%%%%%%%%%%%
\item[rm] 删除文件

\begin{minted}{sh}
# Usage: hadoop fs -rm URI [URI ...]
# Example:
hadoop fs -rm hdfs://nn.example.com/hadoop/hadoopfile /user/hadoop/emptydir

#返回值:0: 正确 -1: 错误
\end{minted}
%%%%%%%%%%%%%%%%%%%%%%%%%%%%%%%%%%%%%%%%%%%%%%%%%%%%%%%%%%%%%%%%%%%%%%%

%%%%%%%%%%%%%%%%%%%%%%%%%%%%%%%%%%%%%%%%%%%%%%%%%%%%%%%%%%%%%%%%%%%%%%%
\item[rmr] 递归删除文件,和UNIX中的 rm -r 差不多

\begin{minted}{sh}
# Usage: hadoop fs -rmr URI [URI ...]
# Example:
hadoop fs -rmr /user/hadoop/dir
hadoop fs -rmr hdfs://nn.example.com/hadoop/hadoopfile /user/hadoop/emptydir

#返回值:0: 正确 -1: 错误
\end{minted}
%%%%%%%%%%%%%%%%%%%%%%%%%%%%%%%%%%%%%%%%%%%%%%%%%%%%%%%%%%%%%%%%%%%%%%%

%%%%%%%%%%%%%%%%%%%%%%%%%%%%%%%%%%%%%%%%%%%%%%%%%%%%%%%%%%%%%%%%%%%%%%%
\item[setrep] 设置文件的复制因子数,说白了就是手动设置文件的副本数,使用-R
可以递归设置一个路径下的所有文件

\begin{minted}{sh}
# Usage: hadoop fs -setrep [-R] <path>
# Example:
hadoop fs -setrep -w 3 -R /user/hadoop/dir1 # -w 参数指的是副本数量

#返回值:0: 正确 -1: 错误
\end{minted}
%%%%%%%%%%%%%%%%%%%%%%%%%%%%%%%%%%%%%%%%%%%%%%%%%%%%%%%%%%%%%%%%%%%%%%%

%%%%%%%%%%%%%%%%%%%%%%%%%%%%%%%%%%%%%%%%%%%%%%%%%%%%%%%%%%%%%%%%%%%%%%%
\item[stat] 查看路径信息

\begin{minted}{sh}
# Usage: hadoop fs -stat URI [URI ...]
# Example:
hadoop fs -stat path

#返回值:0: 正确 -1: 错误
\end{minted}
%%%%%%%%%%%%%%%%%%%%%%%%%%%%%%%%%%%%%%%%%%%%%%%%%%%%%%%%%%%%%%%%%%%%%%%

%%%%%%%%%%%%%%%%%%%%%%%%%%%%%%%%%%%%%%%%%%%%%%%%%%%%%%%%%%%%%%%%%%%%%%%
\item[tail] 查看文件最后字节到标准输出,和UNIX下的tail类似

\begin{minted}{sh}
# Usage: hadoop fs -tail pathname

#返回值:0: 正确 -1: 错误
\end{minted}
%%%%%%%%%%%%%%%%%%%%%%%%%%%%%%%%%%%%%%%%%%%%%%%%%%%%%%%%%%%%%%%%%%%%%%%

%%%%%%%%%%%%%%%%%%%%%%%%%%%%%%%%%%%%%%%%%%%%%%%%%%%%%%%%%%%%%%%%%%%%%%%
\item[test] 查看文件是否存在,是否为0,或者是否是路径,使用-d选项时,
如果返回值是0则是路径,如果返回值是1则是文件,-1则说明路径不存在

\begin{minted}{sh}
# Usage: hadoop fs -test -[ezd] URI
# -e 测试文件是否存在,如果存在返回0
# -z 测试文件是否为空,如果空返回0
# -d 测试文件是否是路径,如果是返回0,如果是文件返回1,其他返回-1,在UNIX返回值上看应该是255
# Example:
hadoop fs -test -e filename

\end{minted}
%%%%%%%%%%%%%%%%%%%%%%%%%%%%%%%%%%%%%%%%%%%%%%%%%%%%%%%%%%%%%%%%%%%%%%%

%%%%%%%%%%%%%%%%%%%%%%%%%%%%%%%%%%%%%%%%%%%%%%%%%%%%%%%%%%%%%%%%%%%%%%%
\item[text] 以文本方式查看文件,支持zip以及TextRecordInputStream[tbd]格式的文件

\begin{minted}{sh}
# Usage: hadoop fs -text <src>

\end{minted}
%%%%%%%%%%%%%%%%%%%%%%%%%%%%%%%%%%%%%%%%%%%%%%%%%%%%%%%%%%%%%%%%%%%%%%%

%%%%%%%%%%%%%%%%%%%%%%%%%%%%%%%%%%%%%%%%%%%%%%%%%%%%%%%%%%%%%%%%%%%%%%%
\item[touchz] 创建个空文件

\begin{minted}{sh}
# Usage: hadoop fs -touchz URI [URI ...]
# Example:
hadoop -touchz pathname

#返回值:0: 正确 -1: 错误
\end{minted}
%%%%%%%%%%%%%%%%%%%%%%%%%%%%%%%%%%%%%%%%%%%%%%%%%%%%%%%%%%%%%%%%%%%%%%%
\end{description}

\subsection{限额管理}
\textit{
如果很多人使用hadoop,设置限额是很有意义的,Hadoop quota的设定是针对路径,而不是针对账号
所以管理上最好每个账号只能写入一个目录,关于权限管理请查看~\ref{pri:1}
}

\begin{itemize}
\item[(1)] 设定方式,设定方式有两种,一种是命令空间\footnote{说白了就是允许你建多少个文件}限定,
另外一种是空间大小设置。
\item[(2)] 预设是没有任何限额设置的,可以使用\code{hadoop fs -count -q pathname} 查看

\begin{minted}{sh}
hadoop fs -count -q /user/hadoop/
# 结果如下,为了能全部显示出来,把tab换成了空格,所以就这样啦,哈哈
# 下面有一个表格,对这几个字段进行详细的解释,请查看下面的表
none inf none inf 2 0 0 hdfs://namenode:9000/user/hadoop
\end{minted}

\begin{center}
	\label{table:1}
        \textit{
        \begin{tabular}{c|c|c|c}
        \hline
        序号 & 文字 & 字段名 & 解释 \\
        \hline
        1 & none & QUOTA & 允许创建文件或文件夹的个数 \\
        \hline
        2 & inf & REMAINING\_QUOTA & 创建文件或文件夹个数剩余 \\
        \hline
        3 & none & SPACE\_QUOTA & 允许创建文件总共大小限制 \\
        \hline
        4 & inf & REMAINING\_SPACE\_QUOTA & 剩余创建文件总共大小限制 \\
        \hline
        5 & 2 & DIR\_COUNT & 路径个数 \\
	\hline
	6 & 0 & FILE\_COUNT & 文件个数 \\
        \hline
	7 & 0 & CONTENT\_SIZE & 总共文件大小 \\
	\hline
        \end{tabular}
        }
\end{center}

计算公式,第一个是命名空间计算公式,第二个为空间计算公式

$$ QUOTA - (DIR\_COUNT + FILE\_COUNT) = REMAINING\_QUOTA $$
$$ SPACE\_QUOTA - CONTENT\_SIZE = REMAINING\_SPACE\_QUOTA $$

下面是一些配置命令的例子:

\begin{minted}{sh}
hadoop fs -count -q /user/hadoop #查看配置限额

# 设置此路径允许创建文件和文件夹的最大数目
# 如果要写入的档案数已经超过设定值,会有错误信息 NSQuotaExceededException
hadoop dfsadmin -setQuota 10000 /user/hadoop/test/ 

# 查看上一步的设定
hadoop fs -count /user/hadoop/test/

# 设置空间最大允许大小,m,g,t分别代表MB,GB,TB
# 如果写入的档案数已经超过目前设定值,则会有:DSQuotaExceededException
hadoop dfsadmin -setSpaceQuota 1g /user/hadoop/test/

# 清除掉设定
hadoop dfsadmin -clrSpaceQuota /user/hadoop/test/
\end{minted}

\item[(3)] 备注:路径改名限额设置依然生效,设定后如果档案量超过预设,依然会存在大小为0
的这些上传文件,这点hadoop做的不咋地,这个可以试验下,或许新版本解决此问题,不得而知,我
没有进行测试。还有其他BUG,这个看你的hadoop版本了,不过到对使用影响不是很大。

\end{itemize}

\subsection{权限管理详解}
\label{pri:1}





\end{document}

