\documentclass{article}
\let\tldocenglish=1  % for live4ht.cfg
\usepackage{tex-live-zh-cn, indentfirst}
\usepackage{graphicx}
\usepackage{color}
\usepackage{xcolor}
\usepackage{verbatim}
\usepackage{comment}
\usepackage{minted}
\usepackage{lettrine}
\definecolor{bg}{rgb}{0.95,0.95,0.95}
% [tbd] means to be development
% [tbr] means to be view

\begin{document}

\title{%
  {\huge \textsf{Java 简明教程}\\\smallskip}%
}

\author{\textit{屈庆磊}\\[2mm]
        \email{quqinglei@icloud.com} 
       }

\date{\textit{2013 年 10 月 9 日}}

\maketitle
\begin{multicols}{2}
\tableofcontents
\end{multicols}

\section{说明}
学习一门语言无怪乎,但又不止于,总之它就是一个工具而已,需要做的就是先宏观上理解、知道结构,然后个个击破
剩下的就是漫长的时间不段的积累。
\begin{itemize}
\item 语法
\item 是面向对象还是面向过程,以及概念
\item 如何使用外部的库,或者称之为包
\item 如何建立自定义库
\item 操作系统相关的一些问题:如线程、锁等
\item 框架,比如\code{Java} 应该熟悉一种编程框架
\item 剩下的都是细节了,学一门语言很简单,而熟悉的使用还是一段比较漫长的路
\end{itemize}

\section{修饰符}
\begin{itemize}
\item \code{public} 的类、类属变量和方法、包内和包外的任何类都可以访问
\item \code{protected} 的类、类属变量和方法,包内的任何类,以及包外的那些继承了此类的子类才能访问
\item \code{private} 的类属变量和方法,包内包外的任何类不能访问
\item 不以以上三种修饰符修饰的类的方法、类属变量都属于\code{friendly}类型,包内的任何类都可以访问它,包
外的任何类都不能访问(包括包外继承此类的子类)
\end{itemize}

\section{包}
%[tbd]
\textit{
和其他语言一样,在Java中也有库的概念,在编写程序的时候也需要调用库,我们也可以自定义库,至于如何建立库,
如何使用我们可以从例子中得到,再好的描述不如一个简单的例子。
}

\textbf{假如我们的当前路径为:\dirname{/home/java} 写一个简单的库文件}
\begin{minted}[linenos, frame=lines, framesep=2mm]{sh}
> mkdir classes # 建立一个路径,自定义的,仅仅作为例子
> cd classes 
> mkdir -p com/quqinglei/tools
> cd com/quqinglei/tools
> touch Name.java
> vi Name.java # 内容如下
\end{minted}

\begin{minted}[linenos, frame=lines, framesep=2mm]{java}
// Name.java
package com.quqinglei.tools;
public class Name {
	public Name() {
		System.out.println("quqinglei");
	}
}
\end{minted}

\textbf{编译库文件,并打包成jar格式的文件,不打包也可以使用}
\begin{minted}[linenos, frame=lines, framesep=2mm]{sh}
// 编译并打包成jar文件
> javac Name.java
> cd ~/classes
> jar -cvf lei.jar com

\end{minted}

\textbf{写一个小程序调用刚才写的库}
\begin{minted}[linenos, frame=lines, framesep=2mm]{sh}
> cd ~
touch Test.java
\end{minted}

\begin{minted}[linenos, frame=lines, framesep=2mm]{java}
// Test.java
import com.quqinglei.tools.*;
public class Test {
	public static void  main(String[] args) {
		Name n = new Name();
	}
}
\end{minted}

\textbf{下面告诉如何编译使用刚才定义的库的小程序,并讲述如何执行}

\begin{minted}[linenos, frame=lines, framesep=2mm]{sh}
# 一下几种方式都可以,如果打成jar包,则可以使用jar包,否则也可以直接
# 使用路径,可以把jar中的路径结构当做路径去理解
# 1
> javac -classpath /home/java/classes/lei.jar:./ Test.java
> java -classpath /home/java/classes/lei.jar:./ Test
quqinglei

# 2
> export CLASSPATH="/home/java/classes/lei.jar:./"
> javac Test.java
> java Test
quqinglei

# 3
> export CLASSPATH="/home/java/classes:./"
> javac Test.java
> java Test
quqinglei

# 4
> javac -classpath /home/java/classes:./ Test.java
> java -classpath /home/java/classes:./ Test
quqinglei
\end{minted}

\textbf{注意:}
\begin{itemize}
\item 库中的类如果被外部引用,构造函数必须是\code{public}声明的
\item 同上,成员函数如果需要被外面的包引用也必须是\code{public}声明的
\end{itemize}


\section{类}
\subsection{其他}
\begin{itemize}
\item 子类继承父类的构造函数,并且可以拥有自己的构造函数,两者是叠加,没有冲突
\item 子类如果重写了某方法,父类的此方法会被废弃,称之为重写,而非叠加
\end{itemize}

\subsection{初始化和清除}
\textbf{
其实本小节讲的就是构造函数,构造函数提供初始化和清除的工作。
}

\begin{itemize}
\item 构造函数可以被重载,非构造函数也是可以重载的
\item 数据类型能从一个较为小的类型自动转变为一个较为大的类型,且靠近较大,而不是最大
\item 数据类型不能从一个大的匹配到构造函数中的小的类型,如果构造函数没有对比此数据类型大的重载构造函数,则
会出现错误,所以尽量不要做一些不匹配的劳什子,精确一点比较好,不要依靠编译器
\item 默认构造函数是没有参数的
\end{itemize}

\subsection{抽象类}
\textbf{
包含抽象方法的类就叫做抽象类,抽象类是一个模板,或者说是接口,只提供抽象,不提供实现方式,抽象类的子类
必须实现抽象类包含的抽象方法。
}
\begin{itemize}
\item 抽象类中不一定包含抽象方法,但包含抽象方法的类一定要被声明成抽象类\code{abstract class ClassName}
\item 抽象类不能用\code{final}修饰,即一个类不能既是最终类又是抽象类
\item \code{abstract}不能与\code{private/static/final/native} 并列修饰同一个方法
\item 抽象类的抽象方法不能拥有代码,它只能是一个声明,如\code{abstract foo();}
\item 抽象类的子类必须定义抽象类的抽象方法
\end{itemize}

\begin{minted}[linenos, frame=lines, framesep=2mm]{text}
\end{minted}
\end{document}
