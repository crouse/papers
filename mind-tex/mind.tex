\documentclass{article}
\let\tldocenglish=1  % for live4ht.cfg

\usepackage{graphicx}
\usepackage{color}
\usepackage{xcolor}
\usepackage{verbatim}
\usepackage{comment}
\usepackage{minted}
\definecolor{bg}{rgb}{0.95,0.95,0.95}
\usepackage{tex-live-zh-cn, indentfirst}

\begin{document}

\title{%
  {\huge \textit{想法与实现}\\\smallskip}%
  {\small \textsf{设计你的想法}}
}

\author{Qu Qinglei 编写 \\[3mm]
        \email{quqinglei@icloud.com}
       }

\date{2013 年 5 月}

\maketitle

\begin{multicols}{2}
\tableofcontents
\label{sec:firstPage}
%\listoftables
\end{multicols}

\section{Raspberry Pi}
\subsection{智能监控装置}
常规的监控装置会把所有的影像都录制下来,虽然很稳妥,但是也耗费大量的空间去存储,同时也
不利于后期的查找,而且没有主动防御的功能。所以我想设计一个新的带有主动防御功能的监控装置
功能如下所述:
\begin{itemize}
\item 采用Raspberry Pi 去开发
\item 使用温度、声音、摄像头、湿度、光、气味传感器赋予计算机超越人类的感知
\item 可以根据外设传感器判断当前的状态
\item 可以判断声音的来源是否为白噪声、汽车、风吹落叶等杂音,并实时记录成文本信息
\item 赋予计算机主动处理一些事件比如警报、警告等
\end{itemize}

\subsection{家庭智能机器人}
设计一个可以擦地板、开窗帘、捡垃圾、管理电视、管理个人事务、记录谈话、沏茶、洗碗、刷马桶、
自消毒、避开障碍物等功能的机器人。机器大脑在主计算机上,自我判断逻辑采用Rasberry Pi建造,
可以在不依赖主计算机存在的情况下处理一些简单事务。

\subsection{智能脑}
我所有设计的设备都被智能脑所控制,智能脑不只是简单的控制,它还有对人的性格和爱好以及心理
进行判断和总结,并在长期形成一个资料库,可以处理一些简单的命令式会话,而非真正意义上的聊天,
其实也可以设计一些聊天思路。
{(\it{返回第\pageref{sec:firstPage}页})}

\section{Network}
\subsection{几页书网站}
做一个互相交流心得的网站,每个人都可以针对一个话题写一个详细的手册,此手册在描述问题,
以及解决方案上要从根本上分析问题。所有手册都提供pdf下载以及在线浏览。另外同样采用评分
机制对每本书进行评价,评价最高的排在此网站此项内容的最前面。
{(\it{返回第\pageref{sec:firstPage}页})}



%\section{简介}
%\label{sec:intro}
%
%\subsection{\TeX\ Live 与 \TeX\ Collection}
%
%本文档描述 \TL{} 软件的主要功能和特性,\TL{} 是 \TeX{} 及其相关程序在
%\acro{GNU}/Linux 及其他类 Unix 系统、\MacOSX\ 和 Windows
%系统下的一套发行版。
%
%你可以直接下载 \TL{},也可以在 \TeX{} 用户组织给会员分发的 \TK{}
%\DVD 中找到。第~\ref{sec:intro}~节中,简要地介绍了
%\DVD 的内容。这两套发行版都是用户组织共同协作完成的。
%这篇文档, 主要介绍 \TL{} 本身。
%
%\TL{} 包括了 \TeX{}, \LaTeXe{}, \ConTeXt, \MF, \MP, \BibTeX{}
%等许多可执行程序;种类繁多的宏包、字体和文档,并支持世界上许多
%不同的语言。
%
%文档末尾的第~\ref{sec:intro}~节 (第~\pageref{sec:intro}~页)
%介绍了这一版 \TL{} 的重要改变。
%
%\subsection{本文是一个例子程序}
%下面是关于一个代码排版工具的使用。
%
%\begin{minted}{c}
%#include <stdio.h>
%int main()
%{
%	printf("Hello world!\n");
%	return 0;
%}
%\end{minted}


\end{document}
