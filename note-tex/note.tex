\documentclass{article}
\let\tldocenglish=1
\usepackage{tex-live-zh-cn, indentfirst}
\usepackage{graphicx}
\usepackage{color}
\usepackage{xcolor}
\usepackage{verbatim}
\usepackage{comment}
\usepackage{minted}
\definecolor{bg}{rgb}{0.95,0.95,0.95}

%\begin{minted}{python}
%\end{minted}
%\colorbox{blue}{text}
%\textcolor{blue}{GEO}

\begin{document}

\title{
  {\huge \textsf{读书笔记}\\\smallskip}
  {\small \textit{一些问题的解答}}
}

\author{\textit{屈庆磊}\\[2mm]
        \email{quqinglei@icloud.com} 
       }

\date{\textit{2013 年 5 月}}

\maketitle
\newpage
\begin{multicols}{2}
\tableofcontents
\end{multicols}
\newpage 

\section{汇编语言,王爽著}
\subsection{page: 128, keyword: mov sp,30h question: why 30h?}
\emph{desc}我们将\textit{cs:10~cs:2F}的内存空间当作栈来使用,初始状态下栈为空,所以
\textit{ss:sp}要指向栈底,则设置\textit{ss:sp}指向\textit{cs:30}。如果对这点还有疑惑
建议回头认真复习一下第三章。

\indent\emph{ans}代码中最前面有24个字,也就是48个字节,转换成十六进制就是0x30,则为
空栈的栈顶地址。或者说,\textit{cs:10~cs:2F}栈顶为 \bf{$0x2F + 1 = 0x30$}

\end{document}
