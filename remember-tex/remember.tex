\documentclass{article}
\let\tldocenglish=1  % for live4ht.cfg
\usepackage{tex-live-zh-cn, indentfirst}
\usepackage{graphicx}
\usepackage{color}
\usepackage{xcolor}
\usepackage{verbatim}
\usepackage{comment}
\usepackage{minted}
\definecolor{bg}{rgb}{0.95,0.95,0.95}

\begin{document}

\title{%
  {\huge \textsf{记忆}\\\smallskip}%
  {\small \textit{一些重要的东西}}
}

\author{\textit{屈庆磊}\\[2mm]
        \email{quqinglei@icloud.com} 
       }

\date{\textit{2013 年 5 月}}

\maketitle
\newpage
\begin{multicols}{2}
\tableofcontents
%\listoftables
\end{multicols}
\newpage 

\section{JAVA}
\subsection{javac 可以编译,但执行java报错}
\begin{description}
\item[原因一] 你可能安装了两个JDK,或者存在两个java命令的路径,而执行javac和java所使用的是两个不同的版本,建议
使用\code{where is java} 看一下到底安装了几个,实在不行可以使用绝对路径编译执行。如:\code{/usr/bin/javac xx.java 
 /usr/bin/java xx}。
\end{description}


\end{document}
