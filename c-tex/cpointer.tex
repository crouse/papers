\documentclass{article}
\let\tldocenglish=1  % for live4ht.cfg

\usepackage{graphicx}
\usepackage{color}
\usepackage{xcolor}
\usepackage{verbatim}
\usepackage{comment}
\usepackage{minted}
\definecolor{bg}{rgb}{0.95,0.95,0.95}
\usepackage{tex-live-zh-cn, indentfirst}

\begin{document}

\title{%
  {\huge \textit{C和指针笔记}\\\smallskip}%
  {\LARGE \textsf{学习与反思}}
}

\author{Qu Qinglei 编写 \\[3mm]
        \email{quqinglei@icloud.com}
       }

\date{2013 年 5 月}

\maketitle

\begin{multicols}{2}
\tableofcontents
%\listoftables
\end{multicols}

\section{快速上手}
\subsection{问题}
\subsubsection{1.2}
声明只需要编写一次,这样在以后维护和修改它时都会更容易。同样,声明只编写一次消除了在多份拷贝中出现
写法不一致的机会。

\subsubsection{1.5}
\begin{minted}{c}
scanf("%d %d %s", &quantity, &price, department);
\end{minted}

\subsubsection{1.8, rearrange程序包含下面的语句}
\begin{minted}{c}
while(gets(input) != NULL) { ...
\end{minted}
当一个数组作为函数的参数进行传递时,函数无法知道它的长度。因此gets函数没办法防止一个非常长的输入行
,从而导致\code{input}数组溢出。\code{fgets}函数要求数组的长度作为参数传递给它,因此不存在问题。

\subsection{编程练习}
\end{document}
