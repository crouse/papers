\documentclass{article}
\let\tldocenglish=1  % for live4ht.cfg

\usepackage{graphicx}
\usepackage{color}
\usepackage{xcolor}
\usepackage{verbatim}
\usepackage{comment}
\definecolor{bg}{rgb}{0.95,0.95,0.95}
\usepackage{tex-live-zh-cn, indentfirst}
\usepackage{minted}

\begin{document}

\title{%
  {\huge \textit{Python 问题集}\\\smallskip}%
}

\author{\textit{屈庆磊整理} \\[3mm]
        \email{quqinglei@icloud.com}
       }

\date{2013 年 5 月}

\maketitle

\begin{multicols}{2}
\tableofcontents
%\listoftables
\end{multicols}

\section{单例模式}
\textsf{定义} 简单的说就是一个类的对象在程序中只存在一份,比较难理解,必须从例子入手。

\subsection{一个普通的类}
\begin{listing}[H]
\begin{minted}{python}
# 定义一个类X, 然后接下来看
>>> class X:
...     pass
... 
>>> l = X() # l 是一个类的实例
>>> l.one = 'one' # 我们给l的one成员赋值
>>> l.one # 打印l.one的值
'one'
>>> m = X() # m 是一个新的实例
>>> m.one 
Traceback (most recent call last):
  File "<stdin>", line 1, in <module>
AttributeError: X instance has no attribute 'one'
>>> print l, m
<__main__.X instance at 0xb742e1cc> <__main__.X instance at 0xb742e38c>
# 打印x和y的地址,从上面看是不同的,因为他们属于不同的实例
\end{minted}
\label{normalClass}
\caption{一个普通的类和它的实例}
\end{listing}

\subsection{单例模式示例一}
\begin{listing}[H]
\begin{minted}{python}
>>> class X:
...     pass
... 
>>> X = lambda singleton = X(): singleton
>>> 
>>> x = X()
>>> x.one = 'one'
>>> x.one
'one'
>>> y = X()
>>> y.one
'one'
>>> print x, y
<__main__.X instance at 0xb742e26c> <__main__.X instance at 0xb742e26c>
# 打印x和y的地址,从上面看他们是相同的
\end{minted}
\label{singleton}
\caption{一个使用lambda的单实例的类和它的实例}
\end{listing}

我们从上面的第\pageref{normalClass}页的代码框中可以看出类的所有实例都是那一个。而从第\pageref{singleton}页的
代码框里可以看出,实例的内存地址是一样的,也就说明它们实际上是一个。

\subsection{单例模式示例二}
\begin{listing}[H]
\begin{minted}{python}
>>> def singleton(cls):
...     instance = {}
...     def get_instance():
...             if cls not in instance:
...                     instance[cls] = cls()
...             return instance[cls]
...     return get_instance
... 
>>> @singleton
... class MyClass:
...     pass
... 
>>> a = MyClass()
>>> b = MyClass()
>>> print a, b
<__main__.MyClass instance at 0xb6ac1aac> <__main__.MyClass instance at 0xb6ac1aac>
# 打印x和y的地址,从上面看他们是相同的
\end{minted}
\label{singleton2}
\caption{单例模式示例二,采用PEP318的实现方式}
\end{listing}

\subsection{单例模式示例三}
\begin{listing}[H]
\begin{minted}{python}
class Singleton:
    """
    A non-thread-safe helper class to ease implementing singletons.
    This should be used as a decorator -- not a metaclass -- to the
    class that should be a singleton.

    The decorated class can define one `__init__` function that
    takes only the `self` argument. Other than that, there are
    no restrictions that apply to the decorated class.

    To get the singleton instance, use the `Instance` method. Trying
    to use `__call__` will result in a `TypeError` being raised.

    Limitations: The decorated class cannot be inherited from.

    """

    def __init__(self, decorated):
        self._decorated = decorated

    def Instance(self):
        """
        Returns the singleton instance. Upon its first call, it creates a
        new instance of the decorated class and calls its `__init__` method.
        On all subsequent calls, the already created instance is returned.

        """
        try:
            return self._instance
        except AttributeError:
            self._instance = self._decorated()
            return self._instance

    def __call__(self):
        raise TypeError('Singletons must be accessed through `Instance()`.')

    def __instancecheck__(self, inst):
        return isinstance(inst, self._decorated)


@Singleton
   class Foo:
       def __init__(self):
           print 'Foo created'

   f = Foo() # Error, this isn't how you get the instance of a singleton

   f = Foo.Instance() # Good. Being explicit is in line with the Python Zen
   g = Foo.Instance() # Returns already created instance

   print f is g # True
\end{minted}
\label{singleton2}
\caption{单例模式示例三}
\end{listing}


\section{类型介绍}
\subsection{大体简单的介绍}
\label{sec:type}
\begin{table}[H]
\begin{tabular}{|p{2cm}|p{5cm}|p{5cm}|}
\hline
\textit{类型} & \textit{描述} & \textit{语法示例}\\\hline
str &  A character string: an immutable sequence of Unicode codepoints. & \code{'Wikipedia', "Wikipedia", """Spanning multiple lines"""}\\\hline

bytearray & A mutable sequence of bytes. & \code{bytearray(b'Some ASCII') bytearray(b"Some ASCII") bytearray([119, 105, 107, 105])} \\\hline

bytes & An immutable sequence of bytes. & code{b'Some ASCII' b"Some ASCII" bytes([119, 105, 107, 105])} \\\hline

list & Mutable list, can contain mixed types. & [4.0, 'string', True] \\\hline

tuple & Immutable, can contain mixed types. & (4.0, 'string', True)\\\hline
set, frozenset & Unordered set, contains no duplicates. A frozenset is immutable. & {4.0, 'string', True} frozenset([4.0, 'string', True]) \\\hline

set, frozenset & Unordered set, contains no duplicates. A frozenset is immutable. & {4.0, 'string', True} frozenset([4.0, 'string', True]) \\\hline

dict &  mutable associative array of key and value pairs. & \code{\{'key1': 1.0, 3: False\}} \\\hline

int & An immutable integer of unlimited magnitude.[43] & \code{42} \\\hline

float & An immutable floating point number (system-defined precision). & 3.1415927\\\hline

complex & immutable complex number with real and imaginary parts. & 3+2.7j\\\hline

bool & immutable truth value. & \code{True False}\\\hline
\end{tabular}

\caption{python类型,来自维基百科}
\end{table}

\section{排序}
\subsection{Python 字典排序}
\begin{minted}{python}
>>> dic = {'a': 100, 'b': 50, 'c': 1000}
>>> sorted(dic.items(), key = lambda x: x[1])
[('b', 50), ('a', 100), ('c', 1000)]
\end{minted}


\end{document}
