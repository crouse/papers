\documentclass{article}
\let\tldocenglish=1  % for live4ht.cfg
\usepackage{tex-live-zh-cn, indentfirst}
\usepackage{graphicx}
\usepackage{color}
\usepackage{xcolor}
\usepackage{verbatim}
\usepackage{comment}
\usepackage{minted}
\definecolor{bg}{rgb}{0.95,0.95,0.95}

\begin{document}

\title{%
  {\huge \textsf{Hive 入门}\\\smallskip}%
  {\small \textit{翻译作品}}
}

\author{\textit{屈庆磊}\\[2mm]
        \email{quqinglei@pwrd.com} 
       }

\date{\textit{2013 年 9 月 11 日}}

\maketitle
\begin{multicols}{2}
\tableofcontents
\end{multicols}


\section{安装配置}
\subsection{安装要求}
\textit{
首先要已经安装了Hadoop,本文档不讲述Hadoop的安装方式和使用方法。
}

\begin{itemize}
\item Java 1.6 以及更高版本
\item Hadoop 0.20.x 以及更高版本
\end{itemize}

\subsection{安装稳定版本的HIVE}
\textit{
用户可以从Apache的官方站点下载Hive的稳定版本:\url{http://hive.apache.org/releases.html}
}

\begin{minted}[linenos, frame=lines, framesep=2mm]{sh}
# 1. 解压安装包,hive安装包一般命名为hive-x.y.z 
$ tar -xzvf hive-x.y.z.tar.gz

# 2. 设置环境变量
$ cd hive-x.y.z

$ export HIVE_HOME={{pwd}} # 临时的环境变量,一般我们要写到脚本里或者bashrc 文件里
# 比如我们把hive解压到了/home/hadoop/hive 路径,那么我们可以如下面一样导出环境变量
$ export HIVE_HOME=/home/hadoop/hive-x.y.z
$ export PATH=$PATH:$HIVE_HOME/bin

# 其实以上步骤执行完就算是安装完了,它就是个解压包,无需配置过多
# 至于编译安装,不再介绍,看官网的吧,此文档为自己学习使用,为加深印象
\end{minted}

\section{运行HIVE}
Hive 运行需要以下两个条件:

\begin{itemize}
\item 你必须在你的路径里安装了Hadoop
\item export HADOOP\_HOME=<hadoop-instsall-dir> 或许你已经在安装Hadoop时导出了环境变量
可以使用\code{echo \$HADOOP\_HOME} 进行测试。
\end{itemize}

现在就可以直接使用了,直接在Terminal里面输入:\code{hive} 就可以直接进入hive的命令行了。
你会得到如下所示的命令行输入界面:

\begin{minted}[linenos, frame=lines, framesep=2mm]{sh}
hive>
\end{minted}

\subsection{配置管理概述}
\begin{itemize}
\item Hive 的的默认配置位置为\dirname{<install-dir>/conf/hive-default.xml}
\item 配置文件另是可以通过环境变量\dirname{HIVGE\_CONF\_DIR} 去改变的
\item Log4j\footnote{Log4j 简单的说就是Log for java,即java的日志组件} 
的配置文档在\dirname{<install-dir>/hive-log4j.properties}
\item 配置文件建立在Hadoop的配置之上,并默认继承Hadoop的配置
\item Hive 的配置可以通过以下途径更改:
	\begin{itemize}
	\item 更改\dirname{hive-site.xml} 文件
	\item 通过命令行更改,如:\code{hive -hiveconf x1=y1 -hiveconf x2=y2}
	此命令分别把y1和y2赋值给变量x1和x2
	\item 通过设置环境变量\code{HIVE\_OPTS="hiveconf x1=y1 hiveconf x2=y2"} 可以达到同样的效果。
	\end{itemize}
\item 运行时配置
	\begin{itemize}
	\item Hive 的查询使用了map-reduce,因此查询行为可以通过更改hadoop配置变量来控制
	\item 命令行中的'SET' 可以用来设置hadoop或者hive的配置变量,如下:

\begin{minted}[linenos, frame=lines, framesep=2mm]{sh}
# SET -v 会打印出现在的所有变量信息,不加-v只会打印hive的变量信息
hive> SET mapred.job.tracker=myhost.mycompany.com:50030;
hive> SET -v;
\end{minted}

	\end{itemize}
\end{itemize}

\subsection{Hive, 本地和mapreduce}
Hive 编译器为大部分的查询生成map-reduce作业。这些作业是否提交为Map-Reduce集群取决于变量:\code{mapred.job.tracker}

Hadoop 也提供一个小巧的参数,用来指定作业是运行在集群上还是运行在用户的工作站上,虽然一般来说一个map-reduce集群就
意味着多个节点。当数据比较小的时候,在单台机器上的运行速度反而要比在集群上运行速度快。数据可以直接从HDFS上透明访
问。当然数据量大的时候集群处理肯定比本地运行速度快,必经本地只有一个reducer。

Hive完全支持本地模式运行,下面的参数可以设置其本地运行:

\begin{minted}[linenos, frame=lines, framesep=2mm]{sh}
hive> SET mapred.job.tracker=local;
\end{minted}

另外变量\code{mapred.local.dir}应该指向一个路径,否则会收到一个分配磁盘空间的异常,一般定义为:
\dirname{/tmp/<username>/mapred/local}

我们还可以通过变量\code{hive.exec.mode.local.auto}来控制hive是否能自动执行本地模式。此模式默认是关闭的,如果
打开此模式Hive会分析每一个map-reduce作业的大小,并根据以下参数确定是否执行本地模式:

\begin{itemize}
\item 输入数据量小于参数:\code{hive.exec.mode.local.auto.inputbytes.max} 的设置值,默认为128MB
\item Map-task数量少于参数:\code{hive.exec.mode.local.auto.tasks.max} 的值,默认为4
\item Reduce-task 应该是1或者0
\end{itemize}









\end{document}



















